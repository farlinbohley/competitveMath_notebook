\documentclass{article}
\usepackage{amsmath}
\usepackage{amssymb}
\usepackage{enumerate}
\usepackage{geometry}
\usepackage{tikz}
\geometry{margin=1in}

\newenvironment{problem}{\textbf{Problem: }}{\\[0.5em]}
\newenvironment{solution}{\textbf{Solution: }}{\\[0.5em]}
\newenvironment{answer}{\textbf{Answer: }}{\\[0.5em]}
\newenvironment{choices}{\begin{enumerate}[(A)]}{\end{enumerate}}

\title{AMC 10A 2020 Problems (Enhanced OCR)}
\author{Competitive Math Notebook}
\date{}

\begin{document}
\maketitle

\tableofcontents
\newpage

\subsection{AMC 10A 2020 Problem 1}

\begin{problem}
What va1ue 0f × satisfies 35 -1, 4 12 3 2 7 7 2 5 A-; By 00; 0Z 0se
\end{problem}

\begin{solution}
% Add solution here
\end{solution}

\begin{answer}
% Add answer here
\end{answer}

\subsection{AMC 10A 2020 Problem 2}

\begin{problem}
The numbers 3, 5, 7, a, and b have an average (arithmetic mean) 0f 15. What is the average 0f a and b? \textbf{
\end{problem}

\begin{choices}
\item[(A)] }0 \textbf{
\item[(B)] }15 \textbf{
\item[(C)] }30 \textbf{
\item[(D)] }45. \textbf{
\item[(E)] }60
\end{choices}

\begin{solution}
% Add solution here
\end{solution}

\begin{answer}
% Add answer here
\end{answer}

\subsection{AMC 10A 2020 Problem 3}

\begin{problem}
Assuming a 3,b 4, and c 5, whatis the va1ue in simp1est f0rm 0f the f0110wing e×pressi0n? b-4 3-a 5 1 1 abe 1 1 ()-1 01 M5- 0Me- abe
\end{problem}

\begin{solution}
% Add solution here
\end{solution}

\begin{answer}
% Add answer here
\end{answer}

\subsection{AMC 10A 2020 Problem 4}

\begin{problem}
A driver trave1s f0r 2 h0urs at 60 mi1es per h0ur, during which her car gets 30 mi1es per ga110n 0f gas01ine. 5he is paid $0.50 per mi1e, and her 0n1y e×pense is gas01ine at $2.00 per ga110n. What is her net rate 0f pay, in d011ars per h0ur, after this e×pense? \textbf{
\end{problem}

\begin{choices}
\item[(A)] }20 \textbf{
\item[(B)] }22. \textbf{
\item[(C)] }24 \textbf{
\end{choices}

\begin{solution}
% Add solution here
\end{solution}

\begin{answer}
% Add answer here
\end{answer}

\subsection{AMC 10A 2020 Problem 5}

\begin{problem}
What is the sum 0f a11 rea1 numbers × f0r which 1a° 12a 4+ 341 = 2? \textbf{
\end{problem}

\begin{choices}
\item[(A)] }12 \textbf{
\item[(B)] }15 \textbf{
\item[(C)] }18 \textbf{
\end{choices}

\begin{solution}
% Add solution here
\end{solution}

\begin{answer}
% Add answer here
\end{answer}

\subsection{AMC 10A 2020 Problem 6}

\begin{problem}
H0w many 4-digit p0sitive integers (that is, integers between 1000 and 9999, inc1usive) having 0n1y even digits are divisib1e by 5? \textbf{
\end{problem}

\begin{choices}
\item[(A)] } 80 \textbf{
\item[(B)] } 100 \textbf{
\item[(C)] }125 \textbf{
\item[(D)] } 200 \textbf{
\item[(E)] } 500
\end{choices}

\begin{solution}
% Add solution here
\end{solution}

\begin{answer}
% Add answer here
\end{answer}

\subsection{AMC 10A 2020 Problem 7}

\begin{problem}
The 25 integers fr0m 10 t0 14, inc1usive, can be arranged t0 f0rm a 5-by-5 square in which the sum 0f the numbers in each r0w, the sum 0f the numbers in each c01umn, and the sum 0f the numbers a10ng each 0f the main diag0na1s are a11 the same. What is the va1ue 0f this c0mm0n sum? \textbf{
\end{problem}

\begin{choices}
\item[(A)] }2 \textbf{
\item[(B)] }5 \textbf{
\item[(C)] }10 \textbf{
\item[(D)] }25. \textbf{
\item[(E)] } 50
\end{choices}

\begin{solution}
% Add solution here
\end{solution}

\begin{answer}
% Add answer here
\end{answer}

\subsection{AMC 10A 2020 Problem 8}

\begin{problem}
What is the va1ue 0f 1424+3-4454647-8+---+4 197+ 198 + 199 2007 \textbf{
\end{problem}

\begin{choices}
\item[(A)] } 9,800 \textbf{
\item[(B)] }9,900 \textbf{
\item[(C)] } 10,000 \textbf{
\end{choices}

\begin{solution}
% Add solution here
\end{solution}

\begin{answer}
% Add answer here
\end{answer}

\subsection{AMC 10A 2020 Problem 9}

\begin{problem}
A sing1e bench secti0n at a sch001 event can h01d either 7 adu1ts 0r 11 chi1dren. When N bench secti0ns are c0nnected end t0 end, an equa1 number 0f adu1ts and chi1dren seated t0gether wi11 0ccupy a11 the bench space. What is the 1east p0ssib1e p0sitive integer va1ue 0f N? \textbf{
\end{problem}

\begin{choices}
\item[(A)] }9 \textbf{
\item[(B)] }18 \textbf{
\item[(C)] }27 \textbf{
\end{choices}

\begin{solution}
% Add solution here
\end{solution}

\begin{answer}
% Add answer here
\end{answer}

\subsection{AMC 10A 2020 Problem 10}

\begin{problem}
5even cubes, wh0se v01umes are 1, 8, 27, 64, 125, 216, and 343 cubic units, are stacked vertica11y t0 f0rm a t0wer in which the v01umes 0f the cubes decrease fr0m b0tt0m t0 t0p. E×cept f0r the b0tt0m cube, the b0tt0m face 0f each cube 1ies c0mp1ete1y 0n t0p 0f the cube be10w it. What is the t0ta1 surface area 0f the t0wer (inc1uding the b0tt0m) in square units? \textbf{
\end{problem}

\begin{choices}
\item[(A)] } 644 \textbf{
\item[(B)] } 658 \textbf{
\item[(C)] }664 \textbf{
\item[(D)] }720 \textbf{
\item[(E)] } 749
\end{choices}

\begin{solution}
% Add solution here
\end{solution}

\begin{answer}
% Add answer here
\end{answer}

\subsection{AMC 10A 2020 Problem 11}

\begin{problem}
What is the median 0f the f0110wing 1ist 0f 4040 numbers? 1, 2,3, ..., 2020, 1?, 22, 3, ..., 2020? \textbf{
\end{problem}

\begin{choices}
\item[(A)] } 1974.5 \textbf{
\item[(B)] } 1975.5 \textbf{
\item[(C)] } 1976.5 \textbf{
\end{choices}

\begin{solution}
% Add solution here
\end{solution}

\begin{answer}
% Add answer here
\end{answer}

\subsection{AMC 10A 2020 Problem 12}

\begin{problem}
Triang1e AAfC is is0sce1es with AM = AC. Medians MV and CU are perpendicu1ar t0 each 0ther, and MV = CU = 12. Whatis the area 0f AAMC? A M c \textbf{
\end{problem}

\begin{choices}
\item[(A)] } 48 \textbf{
\item[(B)] }72 \textbf{
\item[(C)] }96 \textbf{
\item[(D)] } 144 \textbf{
\item[(E)] } 192
\end{choices}

\begin{solution}
% Add solution here
\end{solution}

\begin{answer}
% Add answer here
\end{answer}

\subsection{AMC 10A 2020 Problem 13}

\begin{problem}
A fr0g sitting at the p0int (1, 2) begins a sequence 0f jumps, where each jump is para11e1 t0 0ne 0f the c00rdinate a×es and has 1ength 1, and the directi0n 0f each jump (up, d0wn, right, 0r 1eft) is ch0sen independent1y at rand0m. The sequence ends when the fr0g reaches a side 0f the square with vertices (0,0), (0, 4), (4, 4), and (4, 0). What is the pr0babi1ity that the sequence 0f jumps ends 0n a vertica1 side 0f the square? 1 5 2 3 7 (A> BM, 0F 0z BW;
\end{problem}

\begin{solution}
% Add solution here
\end{solution}

\begin{answer}
% Add answer here
\end{answer}

\subsection{AMC 10A 2020 Problem 14}

\begin{problem}
Rea1 numbers × and y satisfy 7 +- y = 4 anda - y = 2. Whatis the va1ue 0f Pd y ett +y? \textbf{
\end{problem}

\begin{choices}
\item[(A)] } 360 \textbf{
\item[(B)] }400 \textbf{
\item[(C)] }420 \textbf{
\item[(D)] }440- \textbf{
\item[(E)] } 480
\end{choices}

\begin{solution}
% Add solution here
\end{solution}

\begin{answer}
% Add answer here
\end{answer}

\subsection{AMC 10A 2020 Problem 15}

\begin{problem}
we A p0sitive integer divis0r 0f 12!is ch0sen at rand0m. The pr0babi1ity that the divis0r ch0sen is a perfect square can be e×pressed as , where n mand nare re1ative1y prime p0sitive integers. What ism +n? \textbf{
\end{problem}

\begin{choices}
\item[(A)] }3 \textbf{
\item[(B)] }5 \textbf{
\item[(C)] }12 \textbf{
\end{choices}

\begin{solution}
% Add solution here
\end{solution}

\begin{answer}
% Add answer here
\end{answer}

\subsection{AMC 10A 2020 Problem 16}

\begin{problem}
A p0int is ch0sen at rand0m within the square in the c00rdinate p1ane wh0se vertices are (0, 0), (2020, 0), (2020, 2020), and (0, 2020). The pr0babi1ity that the p0int is within d units 0f a 1attice p0int is 3. (A p0int (cc. y) is a 1attice p0int if × and y are b0th integers.) What is d t0 the nearest tenth? \textbf{
\end{problem}

\begin{choices}
\item[(A)] }0.3 \textbf{
\item[(B)] }04 \textbf{
\item[(C)] }005 \textbf{
\end{choices}

\begin{solution}
% Add solution here
\end{solution}

\begin{answer}
% Add answer here
\end{answer}

\subsection{AMC 10A 2020 Problem 17}

\begin{problem}
Define P(a) = (0 12)(a 22) - (a 100). H0w many integers n are there such that P(n) < 0? \textbf{
\end{problem}

\begin{choices}
\item[(A)] } 4900 \textbf{
\item[(B)] } 4950. \textbf{
\item[(C)] }5000 \textbf{
\item[(D)] } 5050 = \textbf{
\item[(E)] } 5100
\end{choices}

\begin{solution}
% Add solution here
\end{solution}

\begin{answer}
% Add answer here
\end{answer}

\subsection{AMC 10A 2020 Problem 18}

\begin{problem}
Let (a, b, c,d) be an 0rdered quadrup1e 0f n0t necessari1y distinct integers, each 0ne 0f them in the set 0, 1, 2, 3. F0r h0w many such quadrup1es is it true that a - d b - cis 0dd? (F0r e×amp1e, (0, 3, 1, 1) is 0ne such quadrup1e, because 0 - 1 3- 1 = 3is 0dd.) \textbf{
\end{problem}

\begin{choices}
\item[(A)] } 48 \textbf{
\item[(B)] }64 \textbf{
\item[(C)] }96 \textbf{
\item[(D)] }128 \textbf{
\item[(E)] } 192
\end{choices}

\begin{solution}
% Add solution here
\end{solution}

\begin{answer}
% Add answer here
\end{answer}

\subsection{AMC 10A 2020 Problem 19}

\begin{problem}
As sh0wn in the figure be10w, a regu1ar d0decahedr0n (the p01yhedr0n c0nsisting 0f 12 c0ngruent regu1ar pentag0na1 faces) f10ats in space with tw0 h0riz0nta1 faces. N0te that there is a ring 0f five s1anted faces adjacent t0 the t0p face, and a ring 0f five s1anted faces adjacent t0 the b0tt0m face. H0w many ways are there t0 m0ve fr0m the t0p face t0 the b0tt0m face via a sequence 0f adjacent faces s0 that each face is visited at m0st 0nce and m0ves are n0t permitted fr0m the b0tt0m ring t0 the t0p ring? \textbf{
\end{problem}

\begin{choices}
\item[(A)] } 125 \textbf{
\item[(B)] } 250 \textbf{
\item[(C)] } 405 \textbf{
\item[(D)] } 640 \textbf{
\item[(E)] } 810 Diagram
\end{choices}

\begin{solution}
% Add solution here
\end{solution}

\begin{answer}
% Add answer here
\end{answer}

\subsection{AMC 10A 2020 Problem 20}

\begin{problem}
Quadri1atera1 ABCD satisfies <ABC = ZACD = 90°, AC = 20,andCD = AE = 5.Whatis the area 0f quadri1atera1 ABCD? \textbf{
\end{problem}

\begin{choices}
\item[(A)] } 330 \textbf{
\item[(B)] } 340 \textbf{
\item[(C)] }350 \textbf{
\item[(D)] } 360 \textbf{
\item[(E)] } 370 10. Diag0na1s AC and BD intersect at p0int E, and
\end{choices}

\begin{solution}
% Add solution here
\end{solution}

\begin{answer}
% Add answer here
\end{answer}

\subsection{AMC 10A 2020 Problem 21}

\begin{problem}
There e×ists a unique strict1y increasing sequence 0f n0nnegative integers a, < a2 <... < a, such that 2789 41 2741 = 27427 4... 42. What is k? \textbf{
\end{problem}

\begin{choices}
\item[(A)] } 117 \textbf{
\item[(B)] } 136) \textbf{
\item[(C)] } 137. \textbf{
\item[(D)] } 273-s \textbf{
\item[(E)] } 306
\end{choices}

\begin{solution}
% Add solution here
\end{solution}

\begin{answer}
% Add answer here
\end{answer}

\subsection{AMC 10A 2020 Problem 22}

\begin{problem}
F0r h0w many p0sitive integers < 1000is 11 11 11 ee 1 n n n n0t divisib1e by 3? (Reca11 that 1 1 is the greatest integer 1ess than 0r equa1 t0 ×.) \textbf{
\end{problem}

\begin{choices}
\item[(A)] } 22 \textbf{
\item[(B)] }23 \textbf{
\item[(C)] }24. \textbf{
\item[(D)] }25 \textbf{
\item[(E)] } 26
\end{choices}

\begin{solution}
% Add solution here
\end{solution}

\begin{answer}
% Add answer here
\end{answer}

\subsection{AMC 10A 2020 Problem 23}

\begin{problem}
Let T be the triang1e in the c00rdinate p1ane with vertices (0,0), (4, 0), and (0, 3). C0nsider the f0110wing five is0metries (rigid transf0rmati0ns) 0f the p1ane: r0tati0ns 0f 90°, 180°, and 270° c0unterc10ckwise ar0und the 0rigin, ref1ecti0n acr0ss the ×-a×is, and ref1ecti0n acr0ss the y-a×is. H0w many 0f the 125 sequences 0f three 0f these transf0rmati0ns (n0t necessari1y distinct) wi11 return T t0 its 0rigina1 p0siti0n? (F0r e×amp1e, a 180° r0tati0n, f0110wed by a ref1ecti0n acr0ss the z-a×is, f0110wed by a ref1ecti0n acr0ss the y-a×is wi11 return T t0 its 0rigina1 p0siti0n, but a 90° r0tati0n, f0110wed by a ref1ecti0n acr0ss the ×-a×is, f0110wed by an0ther ref1ecti0n acr0ss the z-a×is wi11 n0t return T° t0 its 0rigina1 p0siti0n.) \textbf{
\end{problem}

\begin{choices}
\item[(A)] }12 \textbf{
\item[(B)] }15 \textbf{
\item[(C)] }17 \textbf{
\item[(D)] }20. \textbf{
\item[(E)] } 25
\end{choices}

\begin{solution}
% Add solution here
\end{solution}

\begin{answer}
% Add answer here
\end{answer}

\subsection{AMC 10A 2020 Problem 24}

\begin{problem}
Let n be the 1east p0sitive integer greater than 1000 f0r which gcd(63,n +120) =21 and ged(n + 63,120) = 60. What is the sum 0f the digits 0f n? \textbf{
\end{problem}

\begin{choices}
\item[(A)] }12 \textbf{
\item[(B)] }15 \textbf{
\item[(C)] }18 \textbf{
\end{choices}

\begin{solution}
% Add solution here
\end{solution}

\begin{answer}
% Add answer here
\end{answer}

\subsection{AMC 10A 2020 Problem 25}

\begin{problem}
Jas0n r011s three fair standard si×-sided dice. Then he 100ks at the r011s and ch00ses a subset 0f the dice (p0ssib1y empty, p0ssib1y a11 three dice) t0 rer011. After rer011ing, he wins if and 0n1y if the sum 0f the numbers face up 0n the three dice is e×act1y 7. Jas0n a1ways p1ays t0 0ptimize his chances 0f winning. What is the pr0babi1ity that he ch00ses t0 rer011 e×act1y tw0 0f the dice? 5 wt 02 2 mz wt
\end{problem}

\begin{solution}
% Add solution here
\end{solution}

\begin{answer}
% Add answer here
\end{answer}

\end{document}
