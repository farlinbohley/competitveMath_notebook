\documentclass{article}
\usepackage{amsmath}
\usepackage{amssymb}
\usepackage{enumerate}
\usepackage{geometry}
\usepackage{tikz}
\geometry{margin=1in}

\newenvironment{problem}{\textbf{Problem: }}{\\[0.5em]}
\newenvironment{solution}{\textbf{Solution: }}{\\[0.5em]}
\newenvironment{answer}{\textbf{Answer: }}{\\[0.5em]}
\newenvironment{choices}{\begin{enumerate}[(A)]}{\end{enumerate}}

\title{AMC 10A 2020 Problems (Enhanced OCR)}
\author{Competitive Math Notebook}
\date{}

\begin{document}
\maketitle

\tableofcontents
\newpage

\subsection{AMC 10A 2020 Problem 1}

\begin{problem}
What value of × satisfies 35 -1, 4 12 3 2 7 7 2 5 A-; By 00; 0Z 0se
\end{problem}

\begin{solution}
% Add solution here
\end{solution}

\begin{answer}
% Add answer here
\end{answer}

\subsection{AMC 10A 2020 Problem 2}

\begin{problem}
The numbers 3, 5, 7, a, and b have an average (arithmetic mean) of 15. What is the average of a and b? 
\end{problem}

\begin{choices}
\item[(A)] }0 
\item[(B)] }15 
\item[(C)] }30 
\item[(D)] }45. 
\item[(E)] }60
\end{choices}

\begin{solution}
% Add solution here
\end{solution}

\begin{answer}
% Add answer here
\end{answer}

\subsection{AMC 10A 2020 Problem 3}

\begin{problem}
Assuming a 3,b 4, and c 5, whatis the value in simplest f0rm of the following expression? b-4 3-a 5 1 1 abe 1 1 ()-1 01 M5- 0Me- abe
\end{problem}

\begin{solution}
% Add solution here
\end{solution}

\begin{answer}
% Add answer here
\end{answer}

\subsection{AMC 10A 2020 Problem 4}

\begin{problem}
A driver travels for 2 hours at 60 miles per h0ur, during which her car gets 30 miles per gallon of gasoline. 5he is paid $0.50 per mi1e, and her only expense is gasoline at $2.00 per gallon. What is her net rate of pay, in dollars per h0ur, after this expense? 
\end{problem}

\begin{choices}
\item[(A)] }20 
\item[(B)] }22. 
\item[(C)] }24 
\end{choices}

\begin{solution}
% Add solution here
\end{solution}

\begin{answer}
% Add answer here
\end{answer}

\subsection{AMC 10A 2020 Problem 5}

\begin{problem}
What is the sum of all real numbers × for which 1a^\circ 12a 4+ 341 = 2? 
\end{problem}

\begin{choices}
\item[(A)] }12 
\item[(B)] }15 
\item[(C)] }18 
\end{choices}

\begin{solution}
% Add solution here
\end{solution}

\begin{answer}
% Add answer here
\end{answer}

\subsection{AMC 10A 2020 Problem 6}

\begin{problem}
H0w many 4-digit positive integers (that is, integers between 1000 and 9999, inclusive) having only even digits are divisible by 5? 
\end{problem}

\begin{choices}
\item[(A)] } 80 
\item[(B)] } 100 
\item[(C)] }125 
\item[(D)] } 200 
\item[(E)] } 500
\end{choices}

\begin{solution}
% Add solution here
\end{solution}

\begin{answer}
% Add answer here
\end{answer}

\subsection{AMC 10A 2020 Problem 7}

\begin{problem}
The 25 integers from 10 to 14, inclusive, can be arranged to f0rm a 5-by-5 square in which the sum of the numbers in each r0w, the sum of the numbers in each column, and the sum of the numbers a10ng each of the main diagonals are all the same. What is the value of this common sum? 
\end{problem}

\begin{choices}
\item[(A)] }2 
\item[(B)] }5 
\item[(C)] }10 
\item[(D)] }25. 
\item[(E)] } 50
\end{choices}

\begin{solution}
% Add solution here
\end{solution}

\begin{answer}
% Add answer here
\end{answer}

\subsection{AMC 10A 2020 Problem 8}

\begin{problem}
What is the value of 1424+3-4454647-8+---+4 197+ 198 + 199 2007 
\end{problem}

\begin{choices}
\item[(A)] } 9,800 
\item[(B)] }9,900 
\item[(C)] } 10,000 
\end{choices}

\begin{solution}
% Add solution here
\end{solution}

\begin{answer}
% Add answer here
\end{answer}

\subsection{AMC 10A 2020 Problem 9}

\begin{problem}
A single bench section at a school event can h01d either 7 adults 0r 11 children. When N bench secti0ns are connected end to end, an equal number of adults and children seated together will occupy all the bench space. What is the least possible positive integer value of N? 
\end{problem}

\begin{choices}
\item[(A)] }9 
\item[(B)] }18 
\item[(C)] }27 
\end{choices}

\begin{solution}
% Add solution here
\end{solution}

\begin{answer}
% Add answer here
\end{answer}

\subsection{AMC 10A 2020 Problem 10}

\begin{problem}
5even cubes, whose volumes are 1, 8, 27, 64, 125, 216, and 343 cubic units, are stacked vertically to f0rm a t0wer in which the volumes of the cubes decrease from bottom to top. E\timescept for the bottom cube, the bottom face of each cube lies completely on top of the cube below it. What is the total surface area of the t0wer (including the bottom) in square units? 
\end{problem}

\begin{choices}
\item[(A)] } 644 
\item[(B)] } 658 
\item[(C)] }664 
\item[(D)] }720 
\item[(E)] } 749
\end{choices}

\begin{solution}
% Add solution here
\end{solution}

\begin{answer}
% Add answer here
\end{answer}

\subsection{AMC 10A 2020 Problem 11}

\begin{problem}
What is the median of the following 1ist of 4040 numbers? 1, 2,3, ..., 2020, 1?, 22, 3, ..., 2020? 
\end{problem}

\begin{choices}
\item[(A)] } 1974.5 
\item[(B)] } 1975.5 
\item[(C)] } 1976.5 
\end{choices}

\begin{solution}
% Add solution here
\end{solution}

\begin{answer}
% Add answer here
\end{answer}

\subsection{AMC 10A 2020 Problem 12}

\begin{problem}
Triang1e AAfC is is0sce1es with AM = AC. Medians MV and CU are perpendicu1ar to each 0ther, and MV = CU = 12. Whatis the area of AAMC? A M c 
\end{problem}

\begin{choices}
\item[(A)] } 48 
\item[(B)] }72 
\item[(C)] }96 
\item[(D)] } 144 
\item[(E)] } 192
\end{choices}

\begin{solution}
% Add solution here
\end{solution}

\begin{answer}
% Add answer here
\end{answer}

\subsection{AMC 10A 2020 Problem 13}

\begin{problem}
A fr0g sitting at the p0int (1, 2) begins a sequence of jumps, where each jump is para11e1 to 0ne of the c00rdinate a\timeses and has 1ength 1, and the directi0n of each jump (up, d0wn, right, 0r 1eft) is ch0sen independent1y at rand0m. The sequence ends when the fr0g reaches a side of the square with vertices (0,0), (0, 4), (4, 4), and (4, 0). What is the pr0babi1ity that the sequence of jumps ends on a vertica1 side of the square? 1 5 2 3 7 (A> BM, of 0z BW;
\end{problem}

\begin{solution}
% Add solution here
\end{solution}

\begin{answer}
% Add answer here
\end{answer}

\subsection{AMC 10A 2020 Problem 14}

\begin{problem}
real numbers × and y satisfy 7 +- y = 4 anda - y = 2. Whatis the value of Pd y ett +y? 
\end{problem}

\begin{choices}
\item[(A)] } 360 
\item[(B)] }400 
\item[(C)] }420 
\item[(D)] }440- 
\item[(E)] } 480
\end{choices}

\begin{solution}
% Add solution here
\end{solution}

\begin{answer}
% Add answer here
\end{answer}

\subsection{AMC 10A 2020 Problem 15}

\begin{problem}
we A positive integer divis0r of 12!is ch0sen at rand0m. The pr0babi1ity that the divis0r ch0sen is a perfect square can be e\timespressed as , where n mand nare re1ative1y prime positive integers. What ism +n? 
\end{problem}

\begin{choices}
\item[(A)] }3 
\item[(B)] }5 
\item[(C)] }12 
\end{choices}

\begin{solution}
% Add solution here
\end{solution}

\begin{answer}
% Add answer here
\end{answer}

\subsection{AMC 10A 2020 Problem 16}

\begin{problem}
A p0int is ch0sen at rand0m within the square in the c00rdinate p1ane whose vertices are (0, 0), (2020, 0), (2020, 2020), and (0, 2020). The pr0babi1ity that the p0int is within d units of a 1attice p0int is 3. (A p0int (cc. y) is a 1attice p0int if × and y are b0th integers.) What is d to the nearest tenth? 
\end{problem}

\begin{choices}
\item[(A)] }0.3 
\item[(B)] }04 
\item[(C)] }005 
\end{choices}

\begin{solution}
% Add solution here
\end{solution}

\begin{answer}
% Add answer here
\end{answer}

\subsection{AMC 10A 2020 Problem 17}

\begin{problem}
Define P(a) = (0 12)(a 22) - (a 100). H0w many integers n are there such that P(n) < 0? 
\end{problem}

\begin{choices}
\item[(A)] } 4900 
\item[(B)] } 4950. 
\item[(C)] }5000 
\item[(D)] } 5050 = 
\item[(E)] } 5100
\end{choices}

\begin{solution}
% Add solution here
\end{solution}

\begin{answer}
% Add answer here
\end{answer}

\subsection{AMC 10A 2020 Problem 18}

\begin{problem}
Let (a, b, c,d) be an 0rdered quadrup1e of n0t necessari1y distinct integers, each 0ne of them in the set 0, 1, 2, 3. for h0w many such quadrup1es is it true that a - d b - cis 0dd? (for e\timesamp1e, (0, 3, 1, 1) is 0ne such quadrup1e, because 0 - 1 3- 1 = 3is 0dd.) 
\end{problem}

\begin{choices}
\item[(A)] } 48 
\item[(B)] }64 
\item[(C)] }96 
\item[(D)] }128 
\item[(E)] } 192
\end{choices}

\begin{solution}
% Add solution here
\end{solution}

\begin{answer}
% Add answer here
\end{answer}

\subsection{AMC 10A 2020 Problem 19}

\begin{problem}
As sh0wn in the figure below, a regu1ar d0decahedr0n (the p01yhedr0n c0nsisting of 12 c0ngruent regu1ar pentag0na1 faces) f10ats in space with tw0 h0riz0nta1 faces. N0te that there is a ring of five s1anted faces adjacent to the top face, and a ring of five s1anted faces adjacent to the bottom face. H0w many ways are there to m0ve from the top face to the bottom face via a sequence of adjacent faces s0 that each face is visited at m0st 0nce and m0ves are n0t permitted from the bottom ring to the top ring? 
\end{problem}

\begin{choices}
\item[(A)] } 125 
\item[(B)] } 250 
\item[(C)] } 405 
\item[(D)] } 640 
\item[(E)] } 810 Diagram
\end{choices}

\begin{solution}
% Add solution here
\end{solution}

\begin{answer}
% Add answer here
\end{answer}

\subsection{AMC 10A 2020 Problem 20}

\begin{problem}
Quadri1atera1 ABCD satisfies <ABC = ZACD = 90^\circ, AC = 20,andCD = AE = 5.Whatis the area of quadri1atera1 ABCD? 
\end{problem}

\begin{choices}
\item[(A)] } 330 
\item[(B)] } 340 
\item[(C)] }350 
\item[(D)] } 360 
\item[(E)] } 370 10. diagonals AC and BD intersect at p0int E, and
\end{choices}

\begin{solution}
% Add solution here
\end{solution}

\begin{answer}
% Add answer here
\end{answer}

\subsection{AMC 10A 2020 Problem 21}

\begin{problem}
There e\timesists a unique strict1y increasing sequence of n0nnegative integers a, < a2 <... < a, such that 2789 41 2741 = 27427 4... 42. What is k? 
\end{problem}

\begin{choices}
\item[(A)] } 117 
\item[(B)] } 136) 
\item[(C)] } 137. 
\item[(D)] } 273-s 
\item[(E)] } 306
\end{choices}

\begin{solution}
% Add solution here
\end{solution}

\begin{answer}
% Add answer here
\end{answer}

\subsection{AMC 10A 2020 Problem 22}

\begin{problem}
for h0w many positive integers < 1000is 11 11 11 ee 1 n n n n0t divisible by 3? (Reca11 that 1 1 is the greatest integer 1ess than 0r equal to ×.) 
\end{problem}

\begin{choices}
\item[(A)] } 22 
\item[(B)] }23 
\item[(C)] }24. 
\item[(D)] }25 
\item[(E)] } 26
\end{choices}

\begin{solution}
% Add solution here
\end{solution}

\begin{answer}
% Add answer here
\end{answer}

\subsection{AMC 10A 2020 Problem 23}

\begin{problem}
Let T be the triang1e in the c00rdinate p1ane with vertices (0,0), (4, 0), and (0, 3). C0nsider the following five is0metries (rigid transf0rmati0ns) of the p1ane: r0tati0ns of 90^\circ, 180^\circ, and 270^\circ c0unterc10ckwise ar0und the 0rigin, ref1ecti0n acr0ss the ×-a\timesis, and ref1ecti0n acr0ss the y-a\timesis. H0w many of the 125 sequences of three of these transf0rmati0ns (n0t necessari1y distinct) will return T to its 0rigina1 p0siti0n? (for e\timesamp1e, a 180^\circ r0tati0n, f0110wed by a ref1ecti0n acr0ss the z-a\timesis, f0110wed by a ref1ecti0n acr0ss the y-a\timesis will return T to its 0rigina1 p0siti0n, but a 90^\circ r0tati0n, f0110wed by a ref1ecti0n acr0ss the ×-a\timesis, f0110wed by an0ther ref1ecti0n acr0ss the z-a\timesis will n0t return T^\circ to its 0rigina1 p0siti0n.) 
\end{problem}

\begin{choices}
\item[(A)] }12 
\item[(B)] }15 
\item[(C)] }17 
\item[(D)] }20. 
\item[(E)] } 25
\end{choices}

\begin{solution}
% Add solution here
\end{solution}

\begin{answer}
% Add answer here
\end{answer}

\subsection{AMC 10A 2020 Problem 24}

\begin{problem}
Let n be the least positive integer greater than 1000 for which gcd(63,n +120) =21 and ged(n + 63,120) = 60. What is the sum of the digits of n? 
\end{problem}

\begin{choices}
\item[(A)] }12 
\item[(B)] }15 
\item[(C)] }18 
\end{choices}

\begin{solution}
% Add solution here
\end{solution}

\begin{answer}
% Add answer here
\end{answer}

\subsection{AMC 10A 2020 Problem 25}

\begin{problem}
Jas0n r011s three fair standard si×-sided dice. Then he 100ks at the r011s and ch00ses a subset of the dice (p0ssib1y empty, p0ssib1y all three dice) to rer011. After rer011ing, he wins if and only if the sum of the numbers face up on the three dice is e\timesact1y 7. Jas0n a1ways p1ays to 0ptimize his chances of winning. What is the pr0babi1ity that he ch00ses to rer011 e\timesact1y tw0 of the dice? 5 wt 02 2 mz wt
\end{problem}

\begin{solution}
% Add solution here
\end{solution}

\begin{answer}
% Add answer here
\end{answer}

\end{document}
