\documentclass{article}
\usepackage{amsmath}
\usepackage{amssymb}
\usepackage{enumerate}
\usepackage{geometry}
\usepackage{tikz}
\geometry{margin=1in}

\newenvironment{problem}{\textbf{Problem: }}{\\[0.5em]}
\newenvironment{solution}{\textbf{Solution: }}{\\[0.5em]}
\newenvironment{answer}{\textbf{Answer: }}{\\[0.5em]}
\newenvironment{choices}{\begin{enumerate}[(A)]}{\end{enumerate}}

\title{AMC 10B 2019 Problems (Enhanced OCR)}
\author{Competitive Math Notebook}
\date{}

\begin{document}
\maketitle

\tableofcontents
\newpage

\subsection{AMC 10B 2019 Problem 1}

\begin{problem}
A1icia had tw0 c0ntainers. The first was fu11 of water and the sec0nd was empty. 5he p0ured all the water from the first c0ntainer int0 the sec0nd c0ntainer, at which p0int the sec0nd c0ntainer was 3 fu11 of water. What is the rati0 of the v01ume of the first c0ntainer to the v01ume of the sec0nd c0ntainer? 5 4 7 9 11 A; Br ; 0DF 0T
\end{problem}

\begin{solution}
% Add solution here
\end{solution}

\begin{answer}
% Add answer here
\end{answer}

\subsection{AMC 10B 2019 Problem 2}

\begin{problem}
C0nsider the statement, If n is n0t prime, then n 2is prime. Which of the following va1ues of n is a c0untere\timesamp1e to this statement? 
\end{problem}

\begin{choices}
\item[(A)] }11 
\item[(B)] }15 
\item[(C)] }19 
\end{choices}

\begin{solution}
% Add solution here
\end{solution}

\begin{answer}
% Add answer here
\end{answer}

\subsection{AMC 10B 2019 Problem 3}

\begin{problem}
In a high school with 500 students, 40 of the seni0rs p1ay a musica1 instrument, whi1e 30 of the n0n-seni0rs d0 n0t p1ay a musica1 instrument. In all, 46.8 of the students d0 n0t p1ay a musica1 instrument. H0w many n0n-seni0rs p1ay a musica1 instrument? 
\end{problem}

\begin{choices}
\item[(A)] } 66 
\item[(B)] } 154 
\item[(C)] }186 
\end{choices}

\begin{solution}
% Add solution here
\end{solution}

\begin{answer}
% Add answer here
\end{answer}

\subsection{AMC 10B 2019 Problem 4}

\begin{problem}
all 1ines with equati0n a× + by = c such that a, b, c f0rm an arithmetic pr0gressi0n pass thr0ugh a common p0int. What are the c00rdinates of that p0int? \item[(\1)] \2\item[(\1)] \2\item[(\1)] \2\item[(\1)] \2\item[(\1)] \2\end{problem}

\begin{solution}
% Add solution here
\end{solution}

\begin{answer}
% Add answer here
\end{answer}

\subsection{AMC 10B 2019 Problem 5}

\begin{problem}
Triang1e ABC lies in the first quadrant. P0ints A, B, and C are ref1ected acr0ss the 1ine 7 = 2 to p0ints A, B, and C, respective1y. Assume that n0ne of the vertices of the triang1e 1ie on the 1ine y = s×. Which of the following statements is n0t a1ways true? 
\end{problem}

\begin{choices}
\item[(A)] } Triang1e A BC lies in the first quadrant. 
\item[(B)] } Triang1es ABC and A BC have the same area. 
\item[(C)] } The s10pe of 1ine AA is 1. 
\item[(D)] } The s10pes of 1ines A.A and CC are the same. 
\item[(E)] } Lines AB and A B are perpendicu1ar to each 0ther.
\end{choices}

\begin{solution}
% Add solution here
\end{solution}

\begin{answer}
% Add answer here
\end{answer}

\subsection{AMC 10B 2019 Problem 6}

\begin{problem}
There is a positive integer n such that (n + 1)! + (n + 2)! = n! - 440. What is the sum of the digits of n? 
\end{problem}

\begin{solution}
% Add solution here
\end{solution}

\begin{answer}
% Add answer here
\end{answer}

\subsection{AMC 10B 2019 Problem 7}

\begin{problem}
Each piece of candy in a st0re c0sts a wh01e number of cents. Casper has e\timesact1y en0ugh m0ney to buy either 12 pieces of red candy, 14 pieces of green candy, 15 pieces of b1ue candy, 0r n pieces of purp1e candy. A piece of purp1e candy c0sts 20 cents, What is the sma11est possible value of n? 
\end{problem}

\begin{choices}
\item[(A)] } 18 
\item[(B)] }21. 
\item[(C)] }24 
\item[(D)] }25 
\item[(E)] } 28
\end{choices}

\begin{solution}
% Add solution here
\end{solution}

\begin{answer}
% Add answer here
\end{answer}

\subsection{AMC 10B 2019 Problem 8}

\begin{problem}
The figure below sh0ws a square and f0ur equi1atera1 triang1es, with each triang1e having a side 1ying on a side of the square, such that each triang1e has side 1ength 2 and the third vertices of the triang1es meet at the center of the square. The regi0n inside the square but 0utside the triang1es is shaded. What is the area of the shaded regi0n? 
\end{problem}

\begin{choices}
\item[(A)] }4 
\item[(B)] } 4V3-s 
\item[(C)] } 16 43.
\end{choices}

\begin{solution}
% Add solution here
\end{solution}

\begin{answer}
% Add answer here
\end{answer}

\subsection{AMC 10B 2019 Problem 10}

\begin{problem}
In a given p1ane, p0ints A and B are 10 units apart. H0w many p0ints C are there in the p1ane such that the perimeter of AABC is 50 units and the area of A ABC is 100 square units? 
\end{problem}

\begin{choices}
\item[(A)] }0 
\item[(B)] }2 
\item[(C)] }4. 
\item[(D)] }8 - 
\item[(E)] } infinite1y many
\end{choices}

\begin{solution}
% Add solution here
\end{solution}

\begin{answer}
% Add answer here
\end{answer}

\subsection{AMC 10B 2019 Problem 11}

\begin{problem}
Tw0 jars each c0ntain the same number of marb1es, and every marb1e is either b1ue 0r green. In Jar 1 the rati0 of b1ue to green marb1es is 9 : 1, and the rati0 of b1ue to green marb1es in Jar 2 is 8 : 1. There are 95 green marb1es in all. H0w many m0re b1ue marb1es are in Jar 1 than in Jar 2? 
\end{problem}

\begin{choices}
\item[(A)] }5 
\item[(B)] }10 
\item[(C)] }25 
\end{choices}

\begin{solution}
% Add solution here
\end{solution}

\begin{answer}
% Add answer here
\end{answer}

\subsection{AMC 10B 2019 Problem 12}

\begin{problem}
What is the greatest possible sum of the digits in the base-seven representati0n of a positive integer 1ess than 2019? 
\end{problem}

\begin{choices}
\item[(A)] }11 
\item[(B)] }14 
\item[(C)] }22. 
\item[(D)] }23. 
\item[(E)] } 27
\end{choices}

\begin{solution}
% Add solution here
\end{solution}

\begin{answer}
% Add answer here
\end{answer}

\subsection{AMC 10B 2019 Problem 13}

\begin{problem}
What is the sum of all real numbers × for which the median of the numbers 4, 6, 8, 17, and × is equal to the mean of th0se five numbers? 15 35 (4)-5 00 5 M5 0 >
\end{problem}

\begin{solution}
% Add solution here
\end{solution}

\begin{answer}
% Add answer here
\end{answer}

\subsection{AMC 10B 2019 Problem 14}

\begin{problem}
The base-ten representati0n for 19! is 121, 675, 100, 40. , 832, H00, where T, M, and H den0te digits that are n0t given. What is T+M-+H? 
\end{problem}

\begin{solution}
% Add solution here
\end{solution}

\begin{answer}
% Add answer here
\end{answer}

\subsection{AMC 10B 2019 Problem 15}

\begin{problem}
Right triang1es 7; and 72 have areas of 1 and 2, respective1y. A side of 7) is c0ngruent to a side of T>, and a different side of T) is c0ngruent to a different side of T>, What is the square of the pr0duct of the 1engths of the 0ther (third) side of T; and 72? 28 32 34 \item[(\1)] \2\end{problem}

\begin{solution}
% Add solution here
\end{solution}

\begin{answer}
% Add answer here
\end{answer}

\subsection{AMC 10B 2019 Problem 16}

\begin{problem}
In AABC with a right ang1e at C, p0int D lies in the interi0r of AB and p0int EF lies in the interi0r of BC s0 that AC = CD, DE = EB, and the rati0 AC : DE = 4 : 3. Whatis the rati0 AD : DB? 
\end{problem}

\begin{choices}
\item[(A)] }2:3 
\item[(B)] }2:V5 
\item[(C)] }1:1 
\item[(D)] }3:V5 
\item[(E)] }3:2
\end{choices}

\begin{solution}
% Add solution here
\end{solution}

\begin{answer}
% Add answer here
\end{answer}

\subsection{AMC 10B 2019 Problem 17}

\begin{problem}
A red ba11 and a green ba11 are rand0m1y and independent1y t0ssed int0 bins numbered with the positive integers s0 that for each ba11, the pr0babi1ity that it is t0ssed int0 bin k is 2- for k = 1, 2, 3.... What is the pr0babi1ity that the red ba11 is t0ssed int0 a higher-numbered bin than the green ba11? 1 2 1 3 3 (7 Ws ; WM; Bs
\end{problem}

\begin{solution}
% Add solution here
\end{solution}

\begin{answer}
% Add answer here
\end{answer}

\subsection{AMC 10B 2019 Problem 18}

\begin{problem}
Henry decides 0ne m0rning to d0 a w0rk0ut, and he wa1ks 1 of the way from his h0me to his gym. The gym is 2 ki10meters away from Henrys h0me. At that p0int, he changes his mind and wa1ks 3 of the way from where he is back t0ward h0me. When he reaches that p0int, he changes his mind again and wa1ks 3 of the distance from there back t0ward the gym. If Henry keeps changing his mind when he has wa1ked 3 of the distance t0ward either the gym 0r h0me from the p0int where he 1ast changed his mind, he will get very c10se to wa1king back and f0rth between a p0int A ki10meters from h0me and a p0int B ki10meters from h0me. What is 1A BP 2 6 5 3 (Vz 01 , 0Z Ws
\end{problem}

\begin{solution}
% Add solution here
\end{solution}

\begin{answer}
% Add answer here
\end{answer}

\subsection{AMC 10B 2019 Problem 19}

\begin{problem}
Let 5 be the set of all positive integer divis0rs of 100, 000. H0w many numbers are the pr0duct of tw0 distinct e1ements of 5?. 
\end{problem}

\begin{choices}
\item[(A)] }98 
\item[(B)] }100 
\item[(C)] }117. 
\item[(D)] }119 
\item[(E)] } 121
\end{choices}

\begin{solution}
% Add solution here
\end{solution}

\begin{answer}
% Add answer here
\end{answer}

\subsection{AMC 10B 2019 Problem 20}

\begin{problem}
As sh0wn in the figure, 1ine segment AD is trisected by p0ints B and Cs0 that AB = BC = CD = 2. Three semicirc1es of radius 1, 1a a0s AEB, BFC, and CGD, have their diameters on AD, and are tangent to 1ine EG at E, F, and G, respective1y. A circ1e of radius 2 has its center on F. The area of the regi0n inside the circ1e but 0utside the three semicirc1es, shaded in the figure, can be e\timespressed in the f0rm ; -a- etd, where a. b. c, and d are positive integers and a and b are re1ative1y prime. What is a + b+ c+ d? EB G A D 
\end{problem}

\begin{choices}
\item[(A)] }13. 
\item[(B)] } 14. 
\item[(C)] } 15s 
\end{choices}

\begin{solution}
% Add solution here
\end{solution}

\begin{answer}
% Add answer here
\end{answer}

\subsection{AMC 10B 2019 Problem 21}

\begin{problem}
Debra f1ips a fair c0in repeated1y, keeping track of h0w many heads and h0w many tai1s she has seen in total, unti1 she gets either tw0 heads in a r0w 0r tw0 tai1s in a r0w, at which p0int she st0ps f1ipping. What is the pr0babi1ity that she gets tw0 heads in a r0w but she sees a sec0nd tai1 bef0re she sees a sec0nd head? wt 02 0+ Mt wi
\end{problem}

\begin{solution}
% Add solution here
\end{solution}

\begin{answer}
% Add answer here
\end{answer}

\subsection{AMC 10B 2019 Problem 22}

\begin{problem}
Raashan, 5y1via, and Ted p1ay the following game. Each starts with $1. A be11 rings every 15 sec0nds, at which time each of the p1ayers wh0 current1y have m0ney simu1tane0us1y ch00ses 0ne of the 0ther tw0 p1ayers independent1y and at rand0m and gives $1 to that p1ayer. What is the pr0babi1ity that after the be11 has rung 2019 times, each p1ayer will have $1? (for e\timesamp1e, Raashan and Ted may each decide to give $1 to 5yivia, and 5yivia may decide to give her d011ar to Ted, at which p0int Raashan will have $0, 5y1via will have $2, and Ted will have $1, and that is the end of the first r0und of p1ay. In the sec0nd r0und Rashaan has n0 m0ney to give, but 5y1via and Ted might ch00se each 0ther to give their $1 to, and the h01dings will be the same at the end of the sec0nd r0und.) 1 2 w+ 0+ Mi we
\end{problem}

\begin{solution}
% Add solution here
\end{solution}

\begin{answer}
% Add answer here
\end{answer}

\subsection{AMC 10B 2019 Problem 23}

\begin{problem}
P0ints A = (6, 13) and B = (12, 11)1ie on circ1e win the p1ane. 5upp0se that the tangent 1ines to w at A and B intersect at a p0int on the ×- a\timesis. What is the area of w? 83: 21 85: 43: 877 at ea >- of of 0z
\end{problem}

\begin{solution}
% Add solution here
\end{solution}

\begin{answer}
% Add answer here
\end{answer}

\subsection{AMC 10B 2019 Problem 24}

\begin{problem}
Define a sequence recursive1y by 9 = 5 and , - ti +5r, +4 tm = for all n0nnegative integers n. Let mm be the least positive integer such that Im 544 Be In which of the following interva1s d0es m1ie? 
\end{problem}

\begin{choices}
\item[(A)] } [9,26] 
\item[(B)] } [27,80] 
\item[(C)] } [81,242] 
\item[(D)] } [243,728] 
\item[(E)] } [729, 00)
\end{choices}

\begin{solution}
% Add solution here
\end{solution}

\begin{answer}
% Add answer here
\end{answer}

\subsection{AMC 10B 2019 Problem 25}

\begin{problem}
H0w many sequences of 0s and Is of 1ength 19 are there that begin with a 0, end with a 0, c0ntain n0 tw0 c0nsecutive 0s, and c0ntain n0 three c0nsecutive 1s? 
\end{problem}

\begin{choices}
\item[(A)] } 55 
\item[(B)] }60 
\item[(C)] }65 
\item[(D)] }70 
\item[(E)] }75
\end{choices}

\begin{solution}
% Add solution here
\end{solution}

\begin{answer}
% Add answer here
\end{answer}

\end{document}
