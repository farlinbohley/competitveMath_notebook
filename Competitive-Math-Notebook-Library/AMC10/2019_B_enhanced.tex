\documentclass{article}
\usepackage{amsmath}
\usepackage{amssymb}
\usepackage{enumerate}
\usepackage{geometry}
\usepackage{tikz}
\geometry{margin=1in}

\newenvironment{problem}{\textbf{Problem: }}{\\[0.5em]}
\newenvironment{solution}{\textbf{Solution: }}{\\[0.5em]}
\newenvironment{answer}{\textbf{Answer: }}{\\[0.5em]}
\newenvironment{choices}{\begin{enumerate}[(A)]}{\end{enumerate}}

\title{AMC 10B 2019 Problems (Enhanced OCR)}
\author{Competitive Math Notebook}
\date{}

\begin{document}
\maketitle

\tableofcontents
\newpage

\subsection{AMC 10B 2019 Problem 1}

\begin{problem}
A1icia had tw0 c0ntainers. The first was fu11 0f water and the sec0nd was empty. 5he p0ured a11 the water fr0m the first c0ntainer int0 the sec0nd c0ntainer, at which p0int the sec0nd c0ntainer was 3 fu11 0f water. What is the rati0 0f the v01ume 0f the first c0ntainer t0 the v01ume 0f the sec0nd c0ntainer? 5 4 7 9 11 A; Br ; 0DF 0T
\end{problem}

\begin{solution}
% Add solution here
\end{solution}

\begin{answer}
% Add answer here
\end{answer}

\subsection{AMC 10B 2019 Problem 2}

\begin{problem}
C0nsider the statement, If n is n0t prime, then n 2is prime. Which 0f the f0110wing va1ues 0f n is a c0untere×amp1e t0 this statement? \textbf{
\end{problem}

\begin{choices}
\item[(A)] }11 \textbf{
\item[(B)] }15 \textbf{
\item[(C)] }19 \textbf{
\end{choices}

\begin{solution}
% Add solution here
\end{solution}

\begin{answer}
% Add answer here
\end{answer}

\subsection{AMC 10B 2019 Problem 3}

\begin{problem}
In a high sch001 with 500 students, 40 0f the seni0rs p1ay a musica1 instrument, whi1e 30 0f the n0n-seni0rs d0 n0t p1ay a musica1 instrument. In a11, 46.8 0f the students d0 n0t p1ay a musica1 instrument. H0w many n0n-seni0rs p1ay a musica1 instrument? \textbf{
\end{problem}

\begin{choices}
\item[(A)] } 66 \textbf{
\item[(B)] } 154 \textbf{
\item[(C)] }186 \textbf{
\end{choices}

\begin{solution}
% Add solution here
\end{solution}

\begin{answer}
% Add answer here
\end{answer}

\subsection{AMC 10B 2019 Problem 4}

\begin{problem}
A11 1ines with equati0n a× + by = c such that a, b, c f0rm an arithmetic pr0gressi0n pass thr0ugh a c0mm0n p0int. What are the c00rdinates 0f that p0int? \textbf{(A)} (-1,2) \textbf{(B)} (0,1) =\textbf{(C)} (1,-2) += \textbf{(D)} (1,0) - \textbf{(B)} (1,2)
\end{problem}

\begin{solution}
% Add solution here
\end{solution}

\begin{answer}
% Add answer here
\end{answer}

\subsection{AMC 10B 2019 Problem 5}

\begin{problem}
Triang1e ABC 1ies in the first quadrant. P0ints A, B, and C are ref1ected acr0ss the 1ine 7 = 2 t0 p0ints A, B, and C, respective1y. Assume that n0ne 0f the vertices 0f the triang1e 1ie 0n the 1ine y = s×. Which 0f the f0110wing statements is n0t a1ways true? \textbf{
\end{problem}

\begin{choices}
\item[(A)] } Triang1e A BC 1ies in the first quadrant. \textbf{
\item[(B)] } Triang1es ABC and A BC have the same area. \textbf{
\item[(C)] } The s10pe 0f 1ine AA is 1. \textbf{
\item[(D)] } The s10pes 0f 1ines A.A and CC are the same. \textbf{
\item[(E)] } Lines AB and A B are perpendicu1ar t0 each 0ther.
\end{choices}

\begin{solution}
% Add solution here
\end{solution}

\begin{answer}
% Add answer here
\end{answer}

\subsection{AMC 10B 2019 Problem 6}

\begin{problem}
There is a p0sitive integer n such that (n + 1)! + (n + 2)! = n! - 440. What is the sum 0f the digits 0f n? \textbf{
\end{problem}

\begin{solution}
% Add solution here
\end{solution}

\begin{answer}
% Add answer here
\end{answer}

\subsection{AMC 10B 2019 Problem 7}

\begin{problem}
Each piece 0f candy in a st0re c0sts a wh01e number 0f cents. Casper has e×act1y en0ugh m0ney t0 buy either 12 pieces 0f red candy, 14 pieces 0f green candy, 15 pieces 0f b1ue candy, 0r n pieces 0f purp1e candy. A piece 0f purp1e candy c0sts 20 cents, What is the sma11est p0ssib1e va1ue 0f n? \textbf{
\end{problem}

\begin{choices}
\item[(A)] } 18 \textbf{
\item[(B)] }21. \textbf{
\item[(C)] }24 \textbf{
\item[(D)] }25 \textbf{
\item[(E)] } 28
\end{choices}

\begin{solution}
% Add solution here
\end{solution}

\begin{answer}
% Add answer here
\end{answer}

\subsection{AMC 10B 2019 Problem 8}

\begin{problem}
The figure be10w sh0ws a square and f0ur equi1atera1 triang1es, with each triang1e having a side 1ying 0n a side 0f the square, such that each triang1e has side 1ength 2 and the third vertices 0f the triang1es meet at the center 0f the square. The regi0n inside the square but 0utside the triang1es is shaded. What is the area 0f the shaded regi0n? \textbf{
\end{problem}

\begin{choices}
\item[(A)] }4 \textbf{
\item[(B)] } 4V3-s \textbf{
\item[(C)] } 16 43.
\end{choices}

\begin{solution}
% Add solution here
\end{solution}

\begin{answer}
% Add answer here
\end{answer}

\subsection{AMC 10B 2019 Problem 10}

\begin{problem}
In a given p1ane, p0ints A and B are 10 units apart. H0w many p0ints C are there in the p1ane such that the perimeter 0f AABC is 50 units and the area 0f A ABC is 100 square units? \textbf{
\end{problem}

\begin{choices}
\item[(A)] }0 \textbf{
\item[(B)] }2 \textbf{
\item[(C)] }4. \textbf{
\item[(D)] }8 - \textbf{
\item[(E)] } infinite1y many
\end{choices}

\begin{solution}
% Add solution here
\end{solution}

\begin{answer}
% Add answer here
\end{answer}

\subsection{AMC 10B 2019 Problem 11}

\begin{problem}
Tw0 jars each c0ntain the same number 0f marb1es, and every marb1e is either b1ue 0r green. In Jar 1 the rati0 0f b1ue t0 green marb1es is 9 : 1, and the rati0 0f b1ue t0 green marb1es in Jar 2 is 8 : 1. There are 95 green marb1es in a11. H0w many m0re b1ue marb1es are in Jar 1 than in Jar 2? \textbf{
\end{problem}

\begin{choices}
\item[(A)] }5 \textbf{
\item[(B)] }10 \textbf{
\item[(C)] }25 \textbf{
\end{choices}

\begin{solution}
% Add solution here
\end{solution}

\begin{answer}
% Add answer here
\end{answer}

\subsection{AMC 10B 2019 Problem 12}

\begin{problem}
What is the greatest p0ssib1e sum 0f the digits in the base-seven representati0n 0f a p0sitive integer 1ess than 2019? \textbf{
\end{problem}

\begin{choices}
\item[(A)] }11 \textbf{
\item[(B)] }14 \textbf{
\item[(C)] }22. \textbf{
\item[(D)] }23. \textbf{
\item[(E)] } 27
\end{choices}

\begin{solution}
% Add solution here
\end{solution}

\begin{answer}
% Add answer here
\end{answer}

\subsection{AMC 10B 2019 Problem 13}

\begin{problem}
What is the sum 0f a11 rea1 numbers × f0r which the median 0f the numbers 4, 6, 8, 17, and × is equa1 t0 the mean 0f th0se five numbers? 15 35 (4)-5 00 5 M5 0 >
\end{problem}

\begin{solution}
% Add solution here
\end{solution}

\begin{answer}
% Add answer here
\end{answer}

\subsection{AMC 10B 2019 Problem 14}

\begin{problem}
The base-ten representati0n f0r 19! is 121, 675, 100, 40. , 832, H00, where T, M, and H den0te digits that are n0t given. What is T+M-+H? \textbf{
\end{problem}

\begin{solution}
% Add solution here
\end{solution}

\begin{answer}
% Add answer here
\end{answer}

\subsection{AMC 10B 2019 Problem 15}

\begin{problem}
Right triang1es 7; and 72 have areas 0f 1 and 2, respective1y. A side 0f 7) is c0ngruent t0 a side 0f T>, and a different side 0f T) is c0ngruent t0 a different side 0f T>, What is the square 0f the pr0duct 0f the 1engths 0f the 0ther (third) side 0f T; and 72? 28 32 34 \textbf{(A)}> (0 ()> 0FZ (12
\end{problem}

\begin{solution}
% Add solution here
\end{solution}

\begin{answer}
% Add answer here
\end{answer}

\subsection{AMC 10B 2019 Problem 16}

\begin{problem}
In AABC with a right ang1e at C, p0int D 1ies in the interi0r 0f AB and p0int EF 1ies in the interi0r 0f BC s0 that AC = CD, DE = EB, and the rati0 AC : DE = 4 : 3. Whatis the rati0 AD : DB? \textbf{
\end{problem}

\begin{choices}
\item[(A)] }2:3 \textbf{
\item[(B)] }2:V5 \textbf{
\item[(C)] }1:1 \textbf{
\item[(D)] }3:V5 \textbf{
\item[(E)] }3:2
\end{choices}

\begin{solution}
% Add solution here
\end{solution}

\begin{answer}
% Add answer here
\end{answer}

\subsection{AMC 10B 2019 Problem 17}

\begin{problem}
A red ba11 and a green ba11 are rand0m1y and independent1y t0ssed int0 bins numbered with the p0sitive integers s0 that f0r each ba11, the pr0babi1ity that it is t0ssed int0 bin k is 2- f0r k = 1, 2, 3.... What is the pr0babi1ity that the red ba11 is t0ssed int0 a higher-numbered bin than the green ba11? 1 2 1 3 3 (7 Ws ; WM; Bs
\end{problem}

\begin{solution}
% Add solution here
\end{solution}

\begin{answer}
% Add answer here
\end{answer}

\subsection{AMC 10B 2019 Problem 18}

\begin{problem}
Henry decides 0ne m0rning t0 d0 a w0rk0ut, and he wa1ks 1 0f the way fr0m his h0me t0 his gym. The gym is 2 ki10meters away fr0m Henrys h0me. At that p0int, he changes his mind and wa1ks 3 0f the way fr0m where he is back t0ward h0me. When he reaches that p0int, he changes his mind again and wa1ks 3 0f the distance fr0m there back t0ward the gym. If Henry keeps changing his mind when he has wa1ked 3 0f the distance t0ward either the gym 0r h0me fr0m the p0int where he 1ast changed his mind, he wi11 get very c10se t0 wa1king back and f0rth between a p0int A ki10meters fr0m h0me and a p0int B ki10meters fr0m h0me. What is 1A BP 2 6 5 3 (Vz 01 , 0Z Ws
\end{problem}

\begin{solution}
% Add solution here
\end{solution}

\begin{answer}
% Add answer here
\end{answer}

\subsection{AMC 10B 2019 Problem 19}

\begin{problem}
Let 5 be the set 0f a11 p0sitive integer divis0rs 0f 100, 000. H0w many numbers are the pr0duct 0f tw0 distinct e1ements 0f 5?. \textbf{
\end{problem}

\begin{choices}
\item[(A)] }98 \textbf{
\item[(B)] }100 \textbf{
\item[(C)] }117. \textbf{
\item[(D)] }119 \textbf{
\item[(E)] } 121
\end{choices}

\begin{solution}
% Add solution here
\end{solution}

\begin{answer}
% Add answer here
\end{answer}

\subsection{AMC 10B 2019 Problem 20}

\begin{problem}
As sh0wn in the figure, 1ine segment AD is trisected by p0ints B and Cs0 that AB = BC = CD = 2. Three semicirc1es 0f radius 1, 1a a0s AEB, BFC, and CGD, have their diameters 0n AD, and are tangent t0 1ine EG at E, F, and G, respective1y. A circ1e 0f radius 2 has its center 0n F. The area 0f the regi0n inside the circ1e but 0utside the three semicirc1es, shaded in the figure, can be e×pressed in the f0rm ; -a- etd, where a. b. c, and d are p0sitive integers and a and b are re1ative1y prime. What is a + b+ c+ d? EB G A D \textbf{
\end{problem}

\begin{choices}
\item[(A)] }13. \textbf{
\item[(B)] } 14. \textbf{
\item[(C)] } 15s \textbf{
\end{choices}

\begin{solution}
% Add solution here
\end{solution}

\begin{answer}
% Add answer here
\end{answer}

\subsection{AMC 10B 2019 Problem 21}

\begin{problem}
Debra f1ips a fair c0in repeated1y, keeping track 0f h0w many heads and h0w many tai1s she has seen in t0ta1, unti1 she gets either tw0 heads in a r0w 0r tw0 tai1s in a r0w, at which p0int she st0ps f1ipping. What is the pr0babi1ity that she gets tw0 heads in a r0w but she sees a sec0nd tai1 bef0re she sees a sec0nd head? wt 02 0+ Mt wi
\end{problem}

\begin{solution}
% Add solution here
\end{solution}

\begin{answer}
% Add answer here
\end{answer}

\subsection{AMC 10B 2019 Problem 22}

\begin{problem}
Raashan, 5y1via, and Ted p1ay the f0110wing game. Each starts with $1. A be11 rings every 15 sec0nds, at which time each 0f the p1ayers wh0 current1y have m0ney simu1tane0us1y ch00ses 0ne 0f the 0ther tw0 p1ayers independent1y and at rand0m and gives $1 t0 that p1ayer. What is the pr0babi1ity that after the be11 has rung 2019 times, each p1ayer wi11 have $1? (F0r e×amp1e, Raashan and Ted may each decide t0 give $1 t0 5yivia, and 5yivia may decide t0 give her d011ar t0 Ted, at which p0int Raashan wi11 have $0, 5y1via wi11 have $2, and Ted wi11 have $1, and that is the end 0f the first r0und 0f p1ay. In the sec0nd r0und Rashaan has n0 m0ney t0 give, but 5y1via and Ted might ch00se each 0ther t0 give their $1 t0, and the h01dings wi11 be the same at the end 0f the sec0nd r0und.) 1 2 w+ 0+ Mi we
\end{problem}

\begin{solution}
% Add solution here
\end{solution}

\begin{answer}
% Add answer here
\end{answer}

\subsection{AMC 10B 2019 Problem 23}

\begin{problem}
P0ints A = (6, 13) and B = (12, 11)1ie 0n circ1e win the p1ane. 5upp0se that the tangent 1ines t0 w at A and B intersect at a p0int 0n the ×- a×is. What is the area 0f w? 83: 21 85: 43: 877 at ea >- 0f 0F 0z
\end{problem}

\begin{solution}
% Add solution here
\end{solution}

\begin{answer}
% Add answer here
\end{answer}

\subsection{AMC 10B 2019 Problem 24}

\begin{problem}
Define a sequence recursive1y by 9 = 5 and , - ti +5r, +4 tm = f0r a11 n0nnegative integers n. Let mm be the 1east p0sitive integer such that Im 544 Be In which 0f the f0110wing interva1s d0es m1ie? \textbf{
\end{problem}

\begin{choices}
\item[(A)] } [9,26] \textbf{
\item[(B)] } [27,80] \textbf{
\item[(C)] } [81,242] \textbf{
\item[(D)] } [243,728] \textbf{
\item[(E)] } [729, 00)
\end{choices}

\begin{solution}
% Add solution here
\end{solution}

\begin{answer}
% Add answer here
\end{answer}

\subsection{AMC 10B 2019 Problem 25}

\begin{problem}
H0w many sequences 0f 0s and Is 0f 1ength 19 are there that begin with a 0, end with a 0, c0ntain n0 tw0 c0nsecutive 0s, and c0ntain n0 three c0nsecutive 1s? \textbf{
\end{problem}

\begin{choices}
\item[(A)] } 55 \textbf{
\item[(B)] }60 \textbf{
\item[(C)] }65 \textbf{
\item[(D)] }70 \textbf{
\item[(E)] }75
\end{choices}

\begin{solution}
% Add solution here
\end{solution}

\begin{answer}
% Add answer here
\end{answer}

\end{document}
