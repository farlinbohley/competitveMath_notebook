\documentclass{article}
\usepackage{amsmath}
\usepackage{amssymb}
\usepackage{enumerate}
\usepackage{geometry}
\usepackage{tikz}
\geometry{margin=1in}

\newenvironment{problem}{\textbf{Problem: }}{\\[0.5em]}
\newenvironment{solution}{\textbf{Solution: }}{\\[0.5em]}
\newenvironment{answer}{\textbf{Answer: }}{\\[0.5em]}
\newenvironment{choices}{\begin{enumerate}[(A)]}{\end{enumerate}}

\title{AMC 10A 2018 Problems (Enhanced OCR)}
\author{Competitive Math Notebook}
\date{}

\begin{document}
\maketitle

\tableofcontents
\newpage

\subsection{AMC 10A 2018 Problem 1}

\begin{problem}
What is the value of -1 7 (Cee ee) ee 5 11 8 18 15 A; 0B> 0z 0- Wz
\end{problem}

\begin{solution}
% Add solution here
\end{solution}

\begin{answer}
% Add answer here
\end{answer}

\subsection{AMC 10A 2018 Problem 2}

\begin{problem}
Li1iane has 50 m0re s0da than Jacque1ine, and A1ice has 25 m0re s0da than Jacque1ine. What is the re1ati0nship between the am0unts of s0da that Li1iane and A1ice have? 
\end{problem}

\begin{choices}
\item[(A)] } Li1iane has 20 m0re s0da than A1ice. 
\item[(B)] } Li1iane has 25 m0re s0da than A1ice. 
\item[(C)] } Li1iane has 45 m0re s0da than A1ice. 
\item[(D)] } Li1iane has 75 m0re s0da than A1ice. 
\item[(E)] } Li1iane has 100 m0re s0da than A1ice.
\end{choices}

\begin{solution}
% Add solution here
\end{solution}

\begin{answer}
% Add answer here
\end{answer}

\subsection{AMC 10A 2018 Problem 3}

\begin{problem}
A unit of b100d e\timespires after 10! = 10 - 9- 8--- 1 sec0nds. Yasin d0nates a unit of b100d at n00n of January 1. on what day d0es his unit of b100d e\timespire? 
\end{problem}

\begin{solution}
% Add solution here
\end{solution}

\begin{answer}
% Add answer here
\end{answer}

\subsection{AMC 10A 2018 Problem 4}

\begin{problem}
H0w many ways can a student schedu1e 3 mathematics c0urses a1gebra, ge0metry, and number the0ry in a 6-peri0d day if n0 tw0 mathematics c0urses can be taken in c0nsecutive peri0ds? (What c0urses the student takes during the 0ther 3 peri0ds is of n0 c0ncem here.) 
\end{problem}

\begin{choices}
\item[(A)] }3 
\item[(B)] }6 
\item[(C)] }12 
\end{choices}

\begin{solution}
% Add solution here
\end{solution}

\begin{answer}
% Add answer here
\end{answer}

\subsection{AMC 10A 2018 Problem 5}

\begin{problem}
A1ice, B0b, and Char1ie were on a hike and were w0ndering h0w far away the nearest t0wn was. When A1ice said, We are at least 6 miles away, B0b rep1ied, We are at m0st 5 miles away. Char1ie then remarked, Actua11y the nearest t0wn is at m0st 4 miles away. It turned 0ut that n0ne of the three statements were true. Let d be the distance in miles to the nearest t0wn. Which of the following interva1s is the set of all possible va1ues of d? \item[(\1)] \2\item[(\1)] \2\item[(\1)] \2\item[(\1)] \2\end{problem}

\begin{solution}
% Add solution here
\end{solution}

\begin{answer}
% Add answer here
\end{answer}

\subsection{AMC 10A 2018 Problem 6}

\begin{problem}
5angh0 up10aded a vide0 to a website where viewers can v0te that they 1ike 0r dis1ike a vide0. Each vide0 begins with a sc0re of 0, and the sc0re increases by 1 for each 1ike v0te and decreases by 1 for each dis1ike v0te. At 0ne p0int 5angh0 saw that his vide0 had a sc0re of 90, and that 65 of the v0tes cast on his vide0 were 1ike v0tes. H0w many v0tes had been cast on 5angh0s vide0 at that p0int? 
\end{problem}

\begin{choices}
\item[(A)] } 200 
\item[(B)] }300 
\item[(C)] }400 
\end{choices}

\begin{solution}
% Add solution here
\end{solution}

\begin{answer}
% Add answer here
\end{answer}

\subsection{AMC 10A 2018 Problem 7}

\begin{problem}
for h0w many (n0t necessari1y positive) integer va1ues of nis the value of 4000 - (7) an integer? 
\end{problem}

\begin{solution}
% Add solution here
\end{solution}

\begin{answer}
% Add answer here
\end{answer}

\subsection{AMC 10A 2018 Problem 8}

\begin{problem}
J0e has a c011ecti0n of 23 c0ins, c0nsisting of 5-cent c0ins, 10-cent c0ins, and 25-cent c0ins. He has 3 m0re 10-cent c0ins than 5-cent c0ins, and the total value of his c011ecti0n is 320 cents. H0w many m0re 25-cent c0ins d0es J0e have than 5-cent c0ins? 
\end{problem}

\begin{choices}
\item[(A)] }0 
\item[(B)] }1 
\item[(C)] }4
\end{choices}

\begin{solution}
% Add solution here
\end{solution}

\begin{answer}
% Add answer here
\end{answer}

\subsection{AMC 10A 2018 Problem 9}

\begin{problem}
all of the triang1es in the diagram below are simi1ar to is0sce1es triang1e ABC, in which AB = AC. Each of the 7 sma11est triang1es has area 1, and A ABC has area 40. What is the area of trapez0id DBCE? A rs B Cc 
\end{problem}

\begin{choices}
\item[(A)] } 16 
\item[(B)] }18 
\item[(C)] }20. 
\item[(D)] }22 
\item[(E)] }24
\end{choices}

\begin{solution}
% Add solution here
\end{solution}

\begin{answer}
% Add answer here
\end{answer}

\subsection{AMC 10A 2018 Problem 10}

\begin{problem}
5upp0se that real number = satisfies V49 2? 25-4? =3 What is the value of \div49 ×? + 25 ? 
\end{problem}

\begin{choices}
\item[(A)] }8 
\item[(B)] } V334+8 
\item[(C)] }9 
\end{choices}

\begin{solution}
% Add solution here
\end{solution}

\begin{answer}
% Add answer here
\end{answer}

\subsection{AMC 10A 2018 Problem 11}

\begin{problem}
When 7 fair standard 6-sided dice are thr0wn, the pr0babi1ity that the sum of the numbers on the top faces is 10 can be written as n e where nis a positive integer. What is n? 
\end{problem}

\begin{choices}
\item[(A)] } 42 
\item[(B)] }49 
\item[(C)] }56 
\end{choices}

\begin{solution}
% Add solution here
\end{solution}

\begin{answer}
% Add answer here
\end{answer}

\subsection{AMC 10A 2018 Problem 12}

\begin{problem}
H0w many 0rdered pairs of real numbers (2, y) satisfy the following system of equati0ns? a+3y=3 [1e1 - 1u11 =2 
\end{problem}

\begin{solution}
% Add solution here
\end{solution}

\begin{answer}
% Add answer here
\end{answer}

\subsection{AMC 10A 2018 Problem 13}

\begin{problem}
A paper triang1e with sides of 1engths 3, 4, and 5 inches, as sh0wn, is f01ded s0 that p0int A fa11s on p0int B. What is the 1ength in inches of the crease? B y 1 A 4 Cc 1 7 15 \item[(\1)] \2\end{problem}

\begin{solution}
% Add solution here
\end{solution}

\begin{answer}
% Add answer here
\end{answer}

\subsection{AMC 10A 2018 Problem 14}

\begin{problem}
What is the greatest integer 1ess than 0r equal to 1004 2100, 3^\circ 4 296 
\end{problem}

\begin{choices}
\item[(A)] } 80 
\item[(B)] }81 
\item[(C)] }96 
\end{choices}

\begin{solution}
% Add solution here
\end{solution}

\begin{answer}
% Add answer here
\end{answer}

\subsection{AMC 10A 2018 Problem 15}

\begin{problem}
Tw0 circ1es of radius 5 are e\timesterna11y tangent to each 0ther and are interna11y tangent to a circ1e of radius 13 at p0ints A and B, as sh0wn in the diagram. The distance AB can be written in the f0rm =, where m and 7 are re1ative1y prime positive integers. What is m + n? A B 
\end{problem}

\begin{choices}
\item[(A)] } 21 
\item[(B)] }29. 
\item[(C)] }58 = 
\end{choices}

\begin{solution}
% Add solution here
\end{solution}

\begin{answer}
% Add answer here
\end{answer}

\subsection{AMC 10A 2018 Problem 16}

\begin{problem}
Right triang1e ABC has 1eg 1engths AB = 20 and BC = 21. including AB and BC, h0w many 1ine segments with integer 1ength can be drawn from verte× B to a p0int on hyp0tenuse AC? 
\end{problem}

\begin{solution}
% Add solution here
\end{solution}

\begin{answer}
% Add answer here
\end{answer}

\subsection{AMC 10A 2018 Problem 17}

\begin{problem}
Let 5 be a set of 6 integers taken from {1, 2,..., 12} with the pr0perty that if a and b are e1ements of 5 with a < 6, then b is n0t a mu1tip1e of a. What is the least possible value of an e1ement in 5? 
\end{problem}

\begin{choices}
\item[(A)] }2 
\item[(B)] }3 
\item[(C)] }4 
\item[(D)] }5 
\item[(E)] }7
\end{choices}

\begin{solution}
% Add solution here
\end{solution}

\begin{answer}
% Add answer here
\end{answer}

\subsection{AMC 10A 2018 Problem 18}

\begin{problem}
H0w many n0nnegative integers can be written in the f0rm 7-3 4.06 +3^\circ +05 -3^\circ +04 +34 +05-3^\circ + a2-3? +0) +3! +ay-3^\circ, where a; {1,0,1}f0r0 <i < 7? 
\end{problem}

\begin{choices}
\item[(A)] } 512 
\item[(B)] } 729. 
\item[(C)] } 1094. 
\item[(D)] } 3281-s 
\item[(E)] } 59,048
\end{choices}

\begin{solution}
% Add solution here
\end{solution}

\begin{answer}
% Add answer here
\end{answer}

\subsection{AMC 10A 2018 Problem 19}

\begin{problem}
A number 7. is rand0m1y se1ected from the set {11, 13, 15, 17, 19}, and a number 7 is rand0m1y se1ected from {1999, 2000, 2001, . .., 2018}. What is the pr0babi1ity that 77 has a units digit of 1? 1 1 3 7 2 Az By 0y 0y Wz
\end{problem}

\begin{solution}
% Add solution here
\end{solution}

\begin{answer}
% Add answer here
\end{answer}

\subsection{AMC 10A 2018 Problem 20}

\begin{problem}
A scanning c0de c0nsists of a$7 \times 7$ grid of squares, with s0me of its squares c010red b1ack and the rest c010red white. There must be at least 0ne square of each c010r in this grid of 49 squares. A scanning c0de is ca11ed symmetric if its 100k d0es n0t change when the entire square is r0tated by a mu1tip1e of 90^\circ c0unterc10ckwise ar0und its center, n0r when it is ref1ected acr0ss a 1ine j0ining 0pp0site c0rners 0r a 1ine j0ining midp0ints of 0pp0site sides. What is the total number of possible symmetric scanning c0des? 
\end{problem}

\begin{choices}
\item[(A)] } 510 
\item[(B)] } 1022 
\item[(C)] }8190 
\end{choices}

\begin{solution}
% Add solution here
\end{solution}

\begin{answer}
% Add answer here
\end{answer}

\subsection{AMC 10A 2018 Problem 21}

\begin{problem}
Which of the following describes the set of va1ues of a for which the curves z + y- = a and y = × ain the real ×yp1ane intersect at e\timesact1y 3 p0ints? \item[(\1)] \2\end{problem}

\begin{solution}
% Add solution here
\end{solution}

\begin{answer}
% Add answer here
\end{answer}

\subsection{AMC 10A 2018 Problem 22}

\begin{problem}
Let a, b,c, and d be positive integers such that gcd(a, b) = 24, gcd(b, c) following must be a divis0r of a? 
\end{problem}

\begin{choices}
\item[(A)] }5 
\item[(B)] }7 
\item[(C)] }11 
\item[(D)] }13. 
\end{choices}

\begin{solution}
% Add solution here
\end{solution}

\begin{answer}
% Add answer here
\end{answer}

\subsection{AMC 10A 2018 Problem 23}

\begin{problem}
Farmer Pythag0ras has a fie1d in the shape of a right triang1e. The right triang1es 1egs have 1engths 3 and 4 units. In the c0rner where th0se sides meet at a right ang1e, he 1eaves a sma11 unp1anted square 5 s0 that from the air it 100ks 1ike the right ang1e symb01. The rest of the fie1d is p1anted. The sh0rtest distance from 5 to the hyp0tenuse is 2 units. What fracti0n of the fie1d is p1anted? 25 26 73 145 74 0TF \item[(\1)] \2
\end{problem}

\begin{solution}
% Add solution here
\end{solution}

\begin{answer}
% Add answer here
\end{answer}

\subsection{AMC 10A 2018 Problem 24}

\begin{problem}
Triang1e ABC with AB = 50 and AC = 10has area 120. Let D be the midp0int of AB, and 1et E be the midp0int of AC. The ang1e bisect0r of ZBAC intersects DE and BC at F and G, respective1y. What is the area of quadri1atera1 FD BG? 
\end{problem}

\begin{choices}
\item[(A)] }60 
\item[(B)] }65 
\item[(C)] }70 
\item[(D)] }75 
\item[(E)] } 80
\end{choices}

\begin{solution}
% Add solution here
\end{solution}

\begin{answer}
% Add answer here
\end{answer}

\subsection{AMC 10A 2018 Problem 25}

\begin{problem}
for a positive integer n and n0nzer0 digits a, b, and c, 1et A,, be the n-digit integer each of whose digits is equal to a; 1et B,, be the n-digit integer each of whose digits is equal to b, and 1et C, be the 2n-digit (n0t n-digit) integer each of whose digits is equal to c. What is the greatest possible value of a + b + for which there are at least tw0 va1ues of n such that C,, B,, = A2? 
\end{problem}

\begin{choices}
\item[(A)] }12 
\item[(B)] }14 
\item[(C)] }16 
\end{choices}

\begin{solution}
% Add solution here
\end{solution}

\begin{answer}
% Add answer here
\end{answer}

\end{document}
