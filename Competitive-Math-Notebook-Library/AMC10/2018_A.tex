\documentclass{article}
\usepackage{amsmath}
\usepackage{amssymb}
\usepackage{enumerate}
\usepackage{geometry}
\geometry{margin=1in}

\newenvironment{problem}{\textbf{Problem: }}{\\}
\newenvironment{solution}{\textbf{Solution: }}{\\}
\newenvironment{answer}{\textbf{Answer: }}{\\}
\newenvironment{choices}{\begin{enumerate}[(A)]}{\end{enumerate}}

\title{AMC 10A 2018 Problems}
\author{Competitive Math Notebook}
\date{}

\begin{document}
\maketitle

\subsection{AMC 10A 2018 Problem 1}

\begin{problem}
What is the value of ((@+07 +1)" 4 i)" FI? 5 u 8 18 15 w2 @t ©f Ms we
\end{problem}

\begin{solution}
% Add solution here
\end{solution}

\begin{answer}
% Add answer here
\end{answer}

\subsection{AMC 10A 2018 Problem 2}

\begin{problem}
Liliane has 50% more soda than Jacqueline, and Alice has 25% more soda than Jacqueline. What is the relationship between the amounts of soda that Liliane and Alice have? (A) Liliane has 20% more soda than Alice. (B) Liliane has 25% more soda than Alice. (C) Liliane has 45% more soda than Alice. (D) Liliane has 75% more soda than Alice. (E) Liliane has 100% more soda than Alice.
\end{problem}

\begin{solution}
% Add solution here
\end{solution}

\begin{answer}
% Add answer here
\end{answer}

\subsection{AMC 10A 2018 Problem 3}

\begin{problem}
A unit of blood expires after 10! = 10. 9-8 -- - 1 seconds. Yasin donates a unit of blood at noon of January 1. 0n what day does his unit of blood expire? (A) January 2 (B) Jamary 12 (C) January 22. (D) February 11 (B) February 12
\end{problem}

\begin{solution}
% Add solution here
\end{solution}

\begin{answer}
% Add answer here
\end{answer}

\subsection{AMC 10A 2018 Problem 4}

\begin{problem}
How many ways can a student schedule 3 mathematics courses — algebra, geometry, and number theory — in a 6-period day if no two mathematics courses can be taken in consecutive periods? (What courses the student takes during the other 3 periods is of no concem here.) (A)3 (B)6 (C)12 (D)18 (B24
\end{problem}

\begin{solution}
% Add solution here
\end{solution}

\begin{answer}
% Add answer here
\end{answer}

\subsection{AMC 10A 2018 Problem 5}

\begin{problem}
Alice, Bob, and Charlie were on a hike and were wondering how far away the nearest town was. When Alice said, "We are at least 6 miles away,” Bob replied, "We are at most 5 miles away.” Charlie then remarked, ‘Actually the nearest town is at most 4 miles away It turned out that none of the three statements were true. Let d be the distance in miles to the nearest town. Which of the following intervals is the set of all possible values of d? (A) (0,4) (B) (4,5) (C) (4.6) (D) (5,6) _—(B) (5, 00)
\end{problem}

\begin{solution}
% Add solution here
\end{solution}

\begin{answer}
% Add answer here
\end{answer}

\subsection{AMC 10A 2018 Problem 6}

\begin{problem}
Sangho uploaded a video to a website where viewers can vote that they like or dislike a video. Each video begins with a score of 0, and the score increases by 1 for each like vote and decreases by 1 for each dislike vote. At one point Sangho saw that his video had a score of 90, and that, 65% of the votes cast on his video were like votes. How many votes had been cast on Sangho's video at that point? (A) 200 (B) 300 (C)400 (D)500 (E) 600
\end{problem}

\begin{solution}
% Add solution here
\end{solution}

\begin{answer}
% Add answer here
\end{answer}

\subsection{AMC 10A 2018 Problem 7}

\begin{problem}
For how many (not necessarily positive) integer values of nis the value of 4000 - (2) "an integer? (A)3 (B)4 (C)6 (D)8 (B)9
\end{problem}

\begin{solution}
% Add solution here
\end{solution}

\begin{answer}
% Add answer here
\end{answer}

\subsection{AMC 10A 2018 Problem 8}

\begin{problem}
Joe has a collection of 23 coins, consisting of 5-cent coins, 10-cent coins, and 25-cent coins. He has 3 more 10-cent coins than 5-cent coins, and the total value of his collection is 320 cents. How many more 25-cent coins does Joe have than 5-cent coins? (A)0 (B)l (C)2 (WD)3 (B)4
\end{problem}

\begin{solution}
% Add solution here
\end{solution}

\begin{answer}
% Add answer here
\end{answer}

\subsection{AMC 10A 2018 Problem 9}

\begin{problem}
All of the triangles in the diagram below are similar to isosceles triangle ABC, in which AB = AC. Each of the 7 smallest triangles has area 1, and A ABC has area 40. What is the area of trapezoid DBC E? A ro B c (A)16 (B)18 (C)20. (D) 22. (B) 24
\end{problem}

\begin{solution}
% Add solution here
\end{solution}

\begin{answer}
% Add answer here
\end{answer}

\subsection{AMC 10A 2018 Problem 10}

\begin{problem}
Suppose that real number = satisfies 49 — 2? — 25 —2? =3 What is the value of 49 — x? + 25 — x? (A)8 (B) ¥33+8 (C)9 (D)2V10+4 (EB) 12
\end{problem}

\begin{solution}
% Add solution here
\end{solution}

\begin{answer}
% Add answer here
\end{answer}

\subsection{AMC 10A 2018 Problem 11}

\begin{problem}
When 7 fair standard 6-sided dice are thrown, the probability that the sum of the numbers on the top faces is 10 can be written as n where nis a positive integer. What is 2? (A) 42 (B)49 (C)56 (D)63. (BE) 84
\end{problem}

\begin{solution}
% Add solution here
\end{solution}

\begin{answer}
% Add answer here
\end{answer}

\subsection{AMC 10A 2018 Problem 12}

\begin{problem}
How many ordered pairs of real numbers (:, y) satisfy the following system of equations? nt3y=3 [lel — full =2 (A)1 (B)2 (C)3 (D)4 (B)8
\end{problem}

\begin{solution}
% Add solution here
\end{solution}

\begin{answer}
% Add answer here
\end{answer}

\subsection{AMC 10A 2018 Problem 13}

\begin{problem}
A paper triangle with sides of lengths 3, 4, and 5 inches, as shown, is folded so that point A falls on point B. What is the length in inches of the crease? B 5 3 A 4 c 1 7 15 (Al+5v2 @®v3 ©> M0M = 2
\end{problem}

\begin{solution}
% Add solution here
\end{solution}

\begin{answer}
% Add answer here
\end{answer}

\subsection{AMC 10A 2018 Problem 14}

\begin{problem}
What is the greatest integer less than or equal to ioe 4 2100, 396 + 996 (A) 80 (B)81 (C)96 (D)97_— (E) 625
\end{problem}

\begin{solution}
% Add solution here
\end{solution}

\begin{answer}
% Add answer here
\end{answer}

\subsection{AMC 10A 2018 Problem 15}

\begin{problem}
Two circles of radius 5 are externally tangent to each other and are internally tangent to a circle of radius 13 at points A and B, as shown in the diagram. The distance AB can be written in the form ™, where m and 7 are relatively prime positive integers. What is m + 7? 00 (A) 21. (B)29. (C)58 (D) 69 (BE) 93
\end{problem}

\begin{solution}
% Add solution here
\end{solution}

\begin{answer}
% Add answer here
\end{answer}

\subsection{AMC 10A 2018 Problem 16}

\begin{problem}
Right triangle ABC has leg lengths AB = 20 and BC’ = 21. Including AB and BC’, how many line segments with integer length can be drawn from vertex B to a point on hypotenuse AC? (A)5 (B)8 (C)12 (D)13 (EB) 15
\end{problem}

\begin{solution}
% Add solution here
\end{solution}

\begin{answer}
% Add answer here
\end{answer}

\subsection{AMC 10A 2018 Problem 17}

\begin{problem}
Let S be a set of 6 integers taken from {1, 2,..., 12} with the property that if a and b are elements of S with a < b, then b is not a multiple of a. What is the least possible value of an element in S? (A)2 (B)3 (C)4 (D)5 (B)7
\end{problem}

\begin{solution}
% Add solution here
\end{solution}

\begin{answer}
% Add answer here
\end{answer}

\subsection{AMC 10A 2018 Problem 18}

\begin{problem}
How many nonnegative integers can be written in the form 47 +3" +05 +3° + 053° 444-34 +a5-3° + ay +3? +a) 3! + ay-3°, where a; € {—1,0, 1} for0 <i <7? (A) 512 (B) 729 (C) 1094 = (D) 3281._— (B) 59, 048
\end{problem}

\begin{solution}
% Add solution here
\end{solution}

\begin{answer}
% Add answer here
\end{answer}

\subsection{AMC 10A 2018 Problem 19}

\begin{problem}
A number 7m is randomly selected from the set {11, 13, 15, 17, 19}, and a number 7 is randomly selected from {1999, 2000, 2001, ..., 2018}. What is the probability that 77.” has a units digit of 1? 1 1 3 7 2 AS @B- (0> ML (®er “A; B®, 0F 0r Wz
\end{problem}

\begin{solution}
% Add solution here
\end{solution}

\begin{answer}
% Add answer here
\end{answer}

\subsection{AMC 10A 2018 Problem 20}

\begin{problem}
A scanning code consists of a 7 x 7 grid of squares, with some of its squares colored black and the rest colored white. There must be at least one square of each color in this grid of 49 squares. A scanning code is called symmetric if its look does not change when the entire square is rotated by a multiple of 90° counterclockwise around its center, nor when it is reflected across a line joining opposite corners or a line joining midpoints of opposite sides. What is the total number of possible symmetric scanning codes? (A) 510 (B) 1022 (C)8190 (D)8192 (EB) 65,534
\end{problem}

\begin{solution}
% Add solution here
\end{solution}

\begin{answer}
% Add answer here
\end{answer}

\subsection{AMC 10A 2018 Problem 21}

\begin{problem}
Which of the following describes the set of values of a for which the curves 7* + y° = a” and y = x° — ain the real xry-plane intersect at exactly 3 points? (Aya=+ @i<a<i (art Wa=t a>} a=! Legel a>t a=? ast 4A 4A 9 4 9 Q
\end{problem}

\begin{solution}
% Add solution here
\end{solution}

\begin{answer}
% Add answer here
\end{answer}

\subsection{AMC 10A 2018 Problem 22}

\begin{problem}
Let a, b,c, and d be positive integers such that gcd(a, b) = 24, ged(b,c) = 36, ged(c,d) = 54, and 70 < ged(d, a) < 100. Which of the following must be a divisor of a? (A)5 (B)7 (C)ll (D)13— (B17
\end{problem}

\begin{solution}
% Add solution here
\end{solution}

\begin{answer}
% Add answer here
\end{answer}

\subsection{AMC 10A 2018 Problem 23}

\begin{problem}
Farmer Pythagoras has a field in the shape of a right triangle. The right triangle’s legs have lengths 3 and 4 units. In the corner where those sides meet at a right angle, he leaves a small unplanted square 5’ so that from the air it looks like the right angle symbol. The rest of the field is, planted. The shortest distance from SS to the hypotenuse is 2 units. What fraction of the field is planted? 25 26 73 145 74 YS Bl 0f M0 Ws A> Br; 0- 0F; 0z
\end{problem}

\begin{solution}
% Add solution here
\end{solution}

\begin{answer}
% Add answer here
\end{answer}

\subsection{AMC 10A 2018 Problem 24}

\begin{problem}
Triangle ABC with AB = 50 and AC = 10has area 120. Let D be the midpoint of AB, and let E be the midpoint of AC’. The angle bisector of ZBAC intersects DE and BC at F and G, respectively. What is the area of quadrilateral FD BG? (A) 60 (B)65 (C)70 (D)75 (E) 80
\end{problem}

\begin{solution}
% Add solution here
\end{solution}

\begin{answer}
% Add answer here
\end{answer}

\subsection{AMC 10A 2018 Problem 25}

\begin{problem}
For a positive integer m and nonzero digits a, b, and c, let A,, be the n-digit integer each of whose digits is equal to a; let B,, be the n-digit integer each of whose digits is equal to b, and let C’,, be the 2n-digit (not n-digit) integer each of whose digits is equal to c. What is the greatest possible value of a + b+ c for which there are at least two values of n such that C,, — B,, = A?? (A)12. (B)14 (C)16 (D)18 (BE) 20
\end{problem}

\begin{solution}
% Add solution here
\end{solution}

\begin{answer}
% Add answer here
\end{answer}

\end{document}
