\documentclass{article}
\usepackage{amsmath}
\usepackage{amssymb}
\usepackage{enumerate}
\usepackage{geometry}
\geometry{margin=1in}

\newenvironment{problem}{\textbf{Problem: }}{\\}
\newenvironment{solution}{\textbf{Solution: }}{\\}
\newenvironment{answer}{\textbf{Answer: }}{\\}
\newenvironment{choices}{\begin{enumerate}[(A)]}{\end{enumerate}}

\title{AMC 10B 2019 Problems}
\author{Competitive Math Notebook}
\date{}

\begin{document}
\maketitle

\subsection{AMC 10B 2019 Problem 1}

\begin{problem}
Alicia had two containers. The first was ; full of water and the second was empty. She poured all the water from the first container into the second container, at which point the second container was # full of water. What is the ratio of the volume of the first container to the volume of the second container? 5 4 7 9 ul )2 Bl 0s 0- WH A> Bl 0l 0T WS
\end{problem}

\begin{solution}
% Add solution here
\end{solution}

\begin{answer}
% Add answer here
\end{answer}

\subsection{AMC 10B 2019 Problem 2}

\begin{problem}
Consider the statement, "If n is not prime, then n — 2is prime." Which of the following values of n is a counterexample to this statement? (A)ll (B)15 (C)19 (D)21 = (B) 27
\end{problem}

\begin{solution}
% Add solution here
\end{solution}

\begin{answer}
% Add answer here
\end{answer}

\subsection{AMC 10B 2019 Problem 3}

\begin{problem}
In a high school with 500 students, 40% of the seniors play a musical instrument, while 30% of the non-seniors do not play a musical instrument. In all, 46.8% of the students do not play a musical instrument. How many non-seniors play a musical instrument? (A) 66 (B) 154 (C)186 (D) 220 (E) 266
\end{problem}

\begin{solution}
% Add solution here
\end{solution}

\begin{answer}
% Add answer here
\end{answer}

\subsection{AMC 10B 2019 Problem 4}

\begin{problem}
All lines with equation ax + by = c such that a, b, c form an arithmetic progression pass through a common point. What are the coordinates of that point? (A) (-1,2) =(B) (0,1) = (©) (1,-2) =D) 4,0) ~—s (B) (1,2)
\end{problem}

\begin{solution}
% Add solution here
\end{solution}

\begin{answer}
% Add answer here
\end{answer}

\subsection{AMC 10B 2019 Problem 5}

\begin{problem}
Triangle ABC lies in the first quadrant. Points A, B, and C are reflected across the line y = sr to points A’, B’, and C’, respectively. Assume that none of the vertices of the triangle lie on the line y = «. Which of the following statements is not always true? (A) Triangle A’ B’C’ lies in the first quadrant. (B) Triangles ABC and A’ B’C" have the same area. (C) The slope of line AA’ is —1. (D) The slopes of lines A.A’ and CC’ are the same. (E) Lines AB and A’ B’ are perpendicular to each other.
\end{problem}

\begin{solution}
% Add solution here
\end{solution}

\begin{answer}
% Add answer here
\end{answer}

\subsection{AMC 10B 2019 Problem 6}

\begin{problem}
There is a positive integer n such that (n + 1)! + (nm + 2)! = n!- 440. What is the sum of the digits of n? (A)3 (B)8 (C)10 (D)1ll (E) 12
\end{problem}

\begin{solution}
% Add solution here
\end{solution}

\begin{answer}
% Add answer here
\end{answer}

\subsection{AMC 10B 2019 Problem 7}

\begin{problem}
Each piece of candy in a store costs a whole number of cents. Casper has exactly enough money to buy either 12 pieces of red candy, 14 pieces of green candy, 15 pieces of blue candy, or n pieces of purple candy. A piece of purple candy costs 20 cents. What is the smallest possible value of n? (A)18 (B)21 (C)24 (D)25.— (B) 28
\end{problem}

\begin{solution}
% Add solution here
\end{solution}

\begin{answer}
% Add answer here
\end{answer}

\subsection{AMC 10B 2019 Problem 8}

\begin{problem}
The figure below shows a square and four equilateral triangles, with each triangle having a side lying on a side of the square, such that each triangle has side length 2 and the third vertices of the triangles meet at the center of the square. The region inside the square but outside the triangles is shaded. What is the area of the shaded region? (A)4 (B)12-4V3 (Cc) 3V3.—(D) 4V3_—s (EB) 16 — 43
\end{problem}

\begin{solution}
% Add solution here
\end{solution}

\begin{answer}
% Add answer here
\end{answer}

\subsection{AMC 10B 2019 Problem 10}

\begin{problem}
In a given plane, points A and B are 10 units apart. How many points C are there in the plane such that the perimeter of A ABC is 50 units and the area of A ABC is 100 square units? (A)0 (B)2 (C)4. (D)8 (B) infinitely many
\end{problem}

\begin{solution}
% Add solution here
\end{solution}

\begin{answer}
% Add answer here
\end{answer}

\subsection{AMC 10B 2019 Problem 11}

\begin{problem}
Two jars each contain the same number of marbles, and every marble is either blue or green. In Jar 1 the ratio of blue to green marbles is 9 : 1, and the ratio of blue to green marbles in Jar 2 is 8 : 1. There are 95 green marbles in all. How many more blue marbles are in Jar 1 than in Jar 2? (A)5 (B)10 (C)25 (D)45_ (BE) 50
\end{problem}

\begin{solution}
% Add solution here
\end{solution}

\begin{answer}
% Add answer here
\end{answer}

\subsection{AMC 10B 2019 Problem 12}

\begin{problem}
What is the greatest possible sum of the digits in the base-seven representation of a positive integer less than 2019? (A)ll (B)14 (C)22. -(D) 23. (BE) 27
\end{problem}

\begin{solution}
% Add solution here
\end{solution}

\begin{answer}
% Add answer here
\end{answer}

\subsection{AMC 10B 2019 Problem 13}

\begin{problem}
What is the sum of all real numbers x for which the median of the numbers 4, 6, 8, 17, and x is equal to the mean of those five numbers? 15 35 ()-5 B0 ©5 MS @>
\end{problem}

\begin{solution}
% Add solution here
\end{solution}

\begin{answer}
% Add answer here
\end{answer}

\subsection{AMC 10B 2019 Problem 14}

\begin{problem}
The base-ten representation for 19! is 121, 675, 100, 40.7, 832, H00, where T, M, and H denote digits that are not given. What is T+M+H? (A)3 (BS (LR WU Bi7
\end{problem}

\begin{solution}
% Add solution here
\end{solution}

\begin{answer}
% Add answer here
\end{answer}

\subsection{AMC 10B 2019 Problem 15}

\begin{problem}
Right triangles 7; and 7», have areas of 1 and 2, respectively. A side of 7) is congruent to a side of 7, and a different side of 7) is congruent to a different side of T>. What is the square of the product of the lengths of the other (third) side of T; and 73? 28 32 34 ()> (@)10 (©> 0Z ®12
\end{problem}

\begin{solution}
% Add solution here
\end{solution}

\begin{answer}
% Add answer here
\end{answer}

\subsection{AMC 10B 2019 Problem 16}

\begin{problem}
In AABC with a right angle at C, point D lies in the interior of AB and point F lies in the interior of BC’ so that AC = CD, DE = EB, and the ratio AC : DE = 4 : 3. What is the ratio AD : DB? (A)2:3 (B)2:V¥5 (C)1:1 (D)3:V5 (B)3:2
\end{problem}

\begin{solution}
% Add solution here
\end{solution}

\begin{answer}
% Add answer here
\end{answer}

\subsection{AMC 10B 2019 Problem 17}

\begin{problem}
A red ball and a green ball are randomly and independently tossed into bins numbered with the positive integers so that for each ball, the probability that it is tossed into bin k is 2~* for k = 1, 2, 3.... What is the probability that the red ball is tossed into a higher-numbered bin than the green ball? 1 2 1 3 3 A) = B) = C)s D) = E) = A> BM>= ©, M0M, ®W-z
\end{problem}

\begin{solution}
% Add solution here
\end{solution}

\begin{answer}
% Add answer here
\end{answer}

\subsection{AMC 10B 2019 Problem 18}

\begin{problem}
Henry decides one morning to do a workout, and he walks { of the way from his home to his gym. The gym is 2 kilometers away from Henry's home. At that point, he changes his mind and walks 3 of the way from where he is back toward home. When he reaches that point, he changes his mind again and walks $ of the distance from there back toward the gym. If Henry keeps changing his mind when he has walked # of the distance toward either the gym or home from the point where he last changed his mind, he will get very close to walking back and forth between a point A kilometers from home and a point B kilometers from home. What is 1A — 3? 2 6 5 3 AZ Bl 0r MZ Ws
\end{problem}

\begin{solution}
% Add solution here
\end{solution}

\begin{answer}
% Add answer here
\end{answer}

\subsection{AMC 10B 2019 Problem 19}

\begin{problem}
Let S be the set of all positive integer divisors of 100, 000. How many numbers are the product of two distinct elements of S'? (A)98 (B)100 (C)117 (D)119 (BE) 121
\end{problem}

\begin{solution}
% Add solution here
\end{solution}

\begin{answer}
% Add answer here
\end{answer}

\subsection{AMC 10B 2019 Problem 20}

\begin{problem}
As shown in the figure, line segment AD is trisected by points B and C' so that AB = BC = CD = 2. Three semicircles of radius 1, SSN —— AEB, BFC, and CGD, have their diameters on AD, and are tangent to line EG at E. F,and G, respectively. A circle of radius 2 has its center on F. The area of the region inside the circle but outside the three semicircles, shaded in the figure, can be expressed in the form Gm verd, where a, b,c, and d are positive integers and a and b are relatively prime. What is a + b+ ¢ + d? Dy G A D (A) 13 (B)14. (C)15) «(D)16—s (EB) 17
\end{problem}

\begin{solution}
% Add solution here
\end{solution}

\begin{answer}
% Add answer here
\end{answer}

\subsection{AMC 10B 2019 Problem 21}

\begin{problem}
Debra flips a fair coin repeatedly, keeping track of how many heads and how many tails she has seen in total, until she gets either two heads in a row or two tails in a row, at which point she stops flipping. What is the probability that she gets two heads in a row but she sees a second tall before she sees a second head? 1 1 1 1 1 J)e= B> ©. Ms Ws Ae By 0F 0D 0e
\end{problem}

\begin{solution}
% Add solution here
\end{solution}

\begin{answer}
% Add answer here
\end{answer}

\subsection{AMC 10B 2019 Problem 22}

\begin{problem}
Raashan, Sylvia, and Ted play the following game. Each starts with $1. A bell rings every 15 seconds, at which time each of the players who currently have money simultaneously chooses one of the other two players independently and at random and gives $1 to that player. What is the probability that after the bell has rung 2019 times, each player will have $1? (For example, Raashan and Ted may each decide to give $1 to Sylvia, and Sylvia may decide to give her dollar to Ted, at which point Raashan will have $0, Sylvia will have $2, and Ted will have $1, and that is. the end of the first round of play. In the second round Rashaan has no money to give, but Sylvia and Ted might choose each other to give their $1 to, and the holdings will be the same at the end of the second round.) 1 1 1 1 2 Y= BZ 00, 0s; Hs
\end{problem}

\begin{solution}
% Add solution here
\end{solution}

\begin{answer}
% Add answer here
\end{answer}

\subsection{AMC 10B 2019 Problem 23}

\begin{problem}
Points A = (6, 13) and B = (12, 11) lie on circle w in the plane. Suppose that the tangent lines to w at A and B intersect at a point on the x- axis. What is the area of w? 830 210 850 430 8in A) > B) > Cc) > ‘D) — E) (A) = (B) => (= @) = (E) =
\end{problem}

\begin{solution}
% Add solution here
\end{solution}

\begin{answer}
% Add answer here
\end{answer}

\subsection{AMC 10B 2019 Problem 24}

\begin{problem}
Define a sequence recursively by 9 = 5 and 1 ta + Baty +4 tna = a for all nonnegative integers 7. Let 1 be the least positive integer such that tm S44 aay. In which of the following intervals does 1 lie? (A) [9,26] (B) [27,80] (C) [81,242] — (D) [243,728] — (B) [729, 00)
\end{problem}

\begin{solution}
% Add solution here
\end{solution}

\begin{answer}
% Add answer here
\end{answer}

\subsection{AMC 10B 2019 Problem 25}

\begin{problem}
How many sequences of (is and Is of length 19 are there that begin with a 0, end with a 0, contain no two consecutive 0s, and contain no three consecutive 1s? (A)55 (B)60 (C)65 (D)70 (B)75
\end{problem}

\begin{solution}
% Add solution here
\end{solution}

\begin{answer}
% Add answer here
\end{answer}

\end{document}
